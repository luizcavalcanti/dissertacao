\chapter{Introdução}

Os órgãos de segurança e de patrulhamento ambiental lidam diariamente com missões de alto risco. Essas missões têm grande custo de operação e representam um perigo significativo para os agentes humanos envolvidos. Especificamente na região da floresta amazônica, a área a ser patrulhada possui uma dimensão ímpar: 11 mil quilômetros de fronteiras e 22 mil quilômetros de rios, compondo uma área total de 5.217.423 quilômetros quadrados do que é conhecido como Amazônia Legal, o que corresponde à aproximadamente 61\% do território brasileiro \cite{ibge}.

Com o intuito de minimizar os custos, aumentar a eficiência e mitigar os riscos humanos de diversos tipos de atividades, são empregados Veículos Aéreos Não-Tripulados (VANTs). Trata-se de veículos autônomos ou controlados remotamente, que têm como objetivo a agilidade e discrição no reconhecimento de áreas isoladas ou hostis, antes que uma equipe de agentes humanos precise intervir. Essa abordagem permite que as eventuais ameaças sejam previamente identificadas, e o comando tático possa tomar as melhores decisões sobre quando e como agir na região em questão.

Os VANTs têm sido utilizados em uma miríade de missões de diferentes naturezas, com destaque para o combate a crimes em áreas de fronteira. Na região da Floresta Amazônica, os veículos podem ser utilizados no combate ao tráfico de animais, à extração ilegal de madeira e outros crimes ambientais \cite{silva:2013}.

Por acontecerem em regiões isoladas e de difícil acesso, os responsáveis por essas infrações costumam escapar antes que os agentes de segurança possam agir. A aproximação de humanos alerta os infratores, pois os veículos (barcos, helicópteros e veículos terrestres utilitários) usados na abordagem produzem sons bastante perceptíveis nesses ambientes. Por esses motivos, os VANTs representam uma boa alternativa para possibilitar flagrantes e o planejamento tático dessas missões.

Equipados com câmeras de diversos tipos e alcances, esses veículos são capazes de gerar muitas horas de vídeo a cada missão, tornando a análise simultânea ou posterior do material gerado um grande esforço para as pessoas responsáveis pelo planejamento tático, presentes em salas de controle para onde os vídeos são transferidos ao vivo, ou analisados posteriormente.

\section{Motivação}

A quantidade maciça de dados gerados pelos VANTs leva à necessidade de desenvolvimento de procedimentos capazes de avaliar essa grande quantidade de material e identificar os prováveis pontos de interesse. Um sistema capaz de fazer tal análise seria importante para possibilitar a rápida avaliação do material gerado por uma operação, ou mesmo para que uma quantidade menor de profissionais seja necessária para o acompanhamento em tempo real de diversos VANTs em uma mesma missão. Isso cria uma demanda para a filtragem e classificação desses dados em informações ou alertas relevantes aos agentes humanos que supervisionam esses equipamentos.

Ao provar a eficiência de uma implementação automatizada dessa classificação, um significativo aumento pode ser alcançado na qualidade e agilidade da detecção de anomalias em ambientes silvestres, bem como na redução da margem de erro dos agentes humanos envolvidos. Através de processamento de imagens digitais e reconhecimento automático de padrões, é possível avaliar o material com maior rapidez, e sinalizar os pontos de possível interesse dos profissionais que acompanham as missões.

Baseando-se em anomalias encontradas nas imagens aéreas de regiões silvestres, áreas com relativa previsibilidade de paisagem, é possível buscar padrões e reconhecer ameaças ou interesses que possam passar despercebidas por olhos humanos, especialmente em um material com extensão de horas. Tais anomalias incluem acampamentos de caçadores, pistas clandestinas, estradas, embarcações, ou seja, indícios de presença humana em áreas de selva.

Diante desse contexto, este projeto propõe o uso de técnicas de processamento digital de imagens e de reconhecimento de padrões para detectar anomalias em imagens obtidas por VANTs, de uma região da Floresta Amazônica. A detecção de anomalia refere-se à identificação de objetos estranhos ao padrão normal da floresta.

\section{Objetivos}

\subsection{Gerais}

Este trabalho tem como objetivo propor uma solução computacional que envolva Processamento Digital de Imagens (PDI) e aprendizagem de máquina para o problema de detecção de anomalias em imagens aéreas da floresta amazônica, com a intenção de demonstrar que a identificação automática dessas anomalias pode trazer benefícios, em especial agilidade, ao processo que hoje é completamente dependente de agentes humanos, e produzir elevadas taxas de acerto.

\subsection{Específicos}

\begin{itemize}
    \item Organizar uma base de dados de imagens devidamente rotuladas que sirva de referência para futuros estudos desta problemática.
    \item Investigar e definir métodos para extração e seleção de características mais adequadas ao problema em questão.
    \item Realizar experimentos e apontar o melhor conjunto de técnicas para a detecção de anomalias em imagens aéreas da floresta amazônica.
\end{itemize}

\section{Organização do Documento}

O restante deste documento está organizado da seguinte forma: no capítulo 2 discorremos sobre a fundamentação teórica necessária para o entendimento do trabalho proposto, além de trabalhos relacionados.

No capítulo 3 é descrita a metodologia que será empregada no desenvolvimento deste trabalho de pesquisa.

No capítulo 4 são apresentadas a base de dados e a descrição dos experimentos realizados com os resultados preliminares da segmentação semântica de imagens aéreas de floresta amazônica por tipo de terreno, utilizando diversos classificadores.

No capítulo 5 um cronograma é proposto, o qual descreve as atividades que já foram realizadas e os próximos passos deste trabalho.
