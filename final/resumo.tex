\begin{resumo}
    Durante o patrulhamento de crimes ambientais, o tempo de resposta é um componente muito importante no sucesso das missões. Geralmente as infrações ocorrem em lugares ermos e de difícil acesso, características que dificultam tanto o patrulhamento quanto a ação de agentes de preservação ambiental. Para aumentar a taxa de sucesso das abordagens e reduzir o risco de vidas humanas, veículos aéreos não-tripulados (VANTs) podem ser usados para cobrir grandes áreas de floresta em pouco tempo, sem que sejam percebidos por infratores, permitindo que os órgãos de patrulhamento dessas áreas possam planejar e agir com mais eficiência na repressão a esses crimes. O novo problema gerado por essa abordagem é a enorme quantidade de dados gerada durante essas missões, que muitas vezes compreendem horas de vídeo. A inspeção manual de todo esse material em busca de elementos antrópicos é muito cansativa e propensa a erros. Este trabalho apresenta uma avaliação de técnicas de segmentação de imagens, inspeção de características a serem extraídas, seguido da classificação supervisionada destes segmentos para detecção de elementos antrópicos em imagens aéreas da floresta amazônica. Além da publicação de uma base de dados com cerca de 3.000 imagens e 10.000 segmentos devidamente rotulados, o trabalho investiga diferentes estratégias para classificação de elementos antrópicos, com resultados de precisão acima de 94\% para conjuntos de classificadores unários.

    \vspace{\onelineskip}
    \noindent
    \textbf{Palavras-chaves}: aprendizado de máquina, segmentação, classificação, imagens aéreas, elementos antrópicos
\end{resumo}

\begin{resumo}[Abstract]
    During environmental crimes patrolling, the response time is a very important component for the success of the missions. Generally, infractions occur in remote and hard-access places, characteristics that hinder both the patrolling as well the action of environmental protection agents. To increase the approaches' success rate and reduce the risk of human lives, unmanned aerial vehicles (UAVs) can be used to cover large areas of forest in a short time without being perceived by offenders, allowing the patrolling organs responsible for these areas to plan and act more efficiently in the repression of such crimes. The new problem generated by this approach is the huge amount of data generated during these missions, which often includes hours of video. The manual inspection of all this material in searching for anthropic elements is very tiring and error-prone. This work presents a evaluation of image segmentation techniques, inspections of features to be extracted, followed by a supervised classification of those segments for anthropic element detection in amazon's rain forest aerial images. Besides making publicly available a dataset with more than 3,000 images and 10,000 segments labeled accordingly, this work investigates different strategies for anthropic elements classification, with results over 94\% of precision for one-classe classifiers ensembles.
    \vspace{\onelineskip}
    \noindent
    \textbf{Keywords}: machine learning, segmentation, classification, aerial images, anthropic elements
\end{resumo}
