\begin{resumo}
    Durante o patrulhamento de crimes ambientais, o tempo de resposta é um componente muito importante no sucesso das missões. Geralmente as infrações ocorrem em lugares ermos e de difícil acesso, características que dificultam tanto o patrulhamento quanto a ação de agentes de preservação ambiental. Para aumentar a taxa de sucesso das abordagens e reduzir o risco de vidas humanas, veículos aéreos não-tripulados (VANTs) podem ser usados para cobrir grandes áreas de floresta em pouco tempo, sem que sejam percebidos por infratores, permitindo que os órgãos de patrulhamento dessas áreas possam planejar e agir com mais eficiência na repressão a esses crimes. O novo problema gerado por essa abordagem é a enorme quantidade de dados gerada durante essas missões, que muitas vezes compreendem horas de vídeo. A inspeção manual de todo esse material em busca de anomalias e objetos de interesse é muito cansativa e propensa a erros humanos. Esta pesquisa de mestrado propõe realizar a detecção automática de objetos anômalos ao ambiente natural, tais como acampamentos, pistas, estradas e embarcações em imagens aéreas da floresta amazônica. O objetivo final deste trabalho é utilizar técnicas computacionais para reduzir o tempo de análise e a quantidade de dados que precisam ser avaliados por especialistas humanos.

    \vspace{\onelineskip}
    \noindent
    \textbf{Palavras-chaves}: aprendizado de máquina, segmentação, classificação, imagens aéreas
\end{resumo}

\begin{resumo}[Abstract]
    During environmental crimes patrolling, the response time is a very important component in the success of the missions. Generally, infractions occur in remote and hard-access places, characteristics that hinder both the patrolling as well the action of environmental protection agents. To increase the approaches' success rate and reduce the risk of human lives, unmanned aerial vehicles (UAVs) can be used to cover large areas of forest in a short time without being perceived by offenders, allowing the patrolling organs responsible for these areas to plan and act more efficiently in the repression of such crimes. The new problem generated by this approach is the huge amount of data generated during these missions, which often includes hours of video. The manual inspection of all this material in searching for anomalies and objects of interest is very tiring and prone to human error. This master's research proposes to perform automatic detection of anomalous objects in natural environment, such as campgrounds, trails, roads and boats in aerial images of the Amazon forest. The goal of this work is to apply computational techniques to reduce the time of analysis and the amount of data that needs to be assessed by human experts.
    
    \vspace{\onelineskip}
    \noindent
    \textbf{Keywords}: machine learning, segmentation, classification, aerial images
\end{resumo}
