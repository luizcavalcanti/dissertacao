\chapter{Conclusão e Trabalhos Futuros}\label{cap:conclusao}

Neste trabalho foi apresentado um estudo comparativo entre diversas abordagens de aprendizagem supervisionada para solucionar o problema da detecção de elementos antrópicos em imagens aéreas da floresta amazônica.

Antes que os experimentos fossem concluídos, foi necessário confeccionar uma base de dados de segmentações manuais, realizada por diversos indivíduos. Este conjunto de dados foi essencial para a avaliação não apenas dos métodos de segmentação considerados estado-da-arte aplicados ao problema, mas também para a avaliação da consistência da segmentação humana na coleção de imagens em questão.

Dentre as técnicas de segmentação analisadas, o método SRM, que realiza o agrupamento de pixels em uma região de acordo com a similaridade da intensidade e cor dos pixels de regiões vizinhas, obteve erros de consistência global e local inferiores aos demais. É importante notar também que o algoritmo MSEG, que se utiliza de informações de cor e textura, obteve desempenho aproximado ao do SRM, mas com o tempo de execução cerca de quatorze vezes menor. Isto torna o algoritmo MSEG um forte candidato a substituir a técnica SRM em uma aplicação prática, em que o tempo de processamento da segmentação seja um fator de impacto.

Um subproduto do experimento que pretendeu analisar as técnicas de segmentação é uma base de cerca de 10.000 segmentos, devidamente identificados e rotulados. Este é um importante objetivo deste trabalho, pois, junto com as ferramentas de software desenvolvidas para facilitar o processo de classificação das amostras, esta base de dados possibilita que trabalhos futuros sejam realizados tanto na melhoria dos resultados obtidos quanto na reaplicação em outros temas.

Para a classificação dos segmentos e, por consequência, a detecção de elementos anômalos nas imagens, quatro abordagens de modelagem do problema de aprendizagem de máquina foram avaliadas.

A primeira delas utiliza classificadores multi-classe em uma base com rótulos de cinco classes (floresta, vegetação rasteira, água, terra e elementos antrópicos). Neste experimento o algoritmo \textit{K-Nearest Neighbor} (KNN) com $k=3$ obteve os melhores resultados de todo o trabalho. Apesar dos bons resultados, técnicas de aprendizado preguiçoso (\textit{lazy learning}) como o KNN não criam um modelo de aprendizado sucinto, necessitando de boa parte das amostras de treinamento para realizar a classificação de novas amostras, o que pode ser problemático em um ambiente real, com limitações de tempo de processamento e memória disponível.

Adicionalmente, um problema da abordagem com classificadores multi-classe é a pouca generalização do aprendizado sobre elementos antrópicos, visto que se trata de uma classe pouco previsível. Como os elementos antrópicos fazem parte do modelo de treinamento, novas amostras de objetos antrópicos cujas características se distanciem das amostras de elementos antrópicos da base de treinamento tendem a reduzir a precisão do modelo.

A segunda abordagem analisada, que utiliza classificadores binários, obteve desempenho acima do esperado. A expectativa era de que os modelos gerados por esta abordagem tivessem um baixo desempenho na classificação da classe de elementos antrópicos, uma vez que esta correspondia a apenas 0.3\% da base de dados. Esperava-se também que por, conta da re-classificação das amostras de segmentos que reagrupou as classes de floresta, vegetação rasteira, água e terra em uma única classe, o modelo gerado fosse pouco descritivo, hipótese que não pôde ser comprovada.

Neste experimento o algoritmo KNN também obteve o melhor desempenho para a classe de elementos antrópicos, desta vez se beneficiando da seleção de características feita através da técnica CFS, que utiliza a correlação entre os atributos do problema para reduzir a dimensão do vetor de características. Os resultados indicam que a seleção de atributos contribuiu significativamente para a simplificação do modelo gerado, tornando-o mais generalista e mais apto a descrever a classe de elementos naturais, resultado inverso ao experimento com classificadores multi-classe.

Na terceira abordagem, pudemos avaliar o desempenho de alguns algoritmos de classificação unária, também conhecidos como detectores de anomalias (\textit{outliers}). Assim como na abordagem de classificação binária, aqui foi preciso agrupar as amostras das classes majoritárias em uma única classe de elementos naturais, agravando o desbalanceamento das classes na base de treinamento.

Os resultados foram bastante similares à abordagem binária, com visível melhora dos resultados quando se usa a seleção de atributos através do método CFS. Neste caso, o algoritmo No entanto, ao contrário da abordagem binária, classificadores unários tendem a responder melhor quando amostras não-usuais da classe de \textit{outliers} aparecem.

A última abordagem analisada neste trabalho se utiliza de um conjunto de classificadores unários. Uma vez que cada classificador é responsável por aprender e criar um modelo para reconhecer uma classe do problema, esta estratégia permitiu que modelos mais especializados fossem criados e que diferentes maneiras de chegar a um consenso sobre a classificação das amostras fossem testados.

Com bons resultados desde o início dos experimentos, o conjunto de classificadores unários se mostra uma abordagem com boa conciliação entre a capacidade de classificar corretamente as amostras de interesse e a capacidade de generalização e detecção de elementos antrópicos ausentes da base de treinamento.

Como métrica para avaliação das abordagens de aprendizado ficou definido que a medida $F1$, que leva em consideração um balanço entre precisão e revocação, seria utilizada para julgar o desempenho dos algoritmos. Por conta da importância da classe, bem como forma de compensar o desequilíbrio na distribuição das classes do problema na base de dados de treinamento, a medida $F1$ da classe de elementos antrópicos foi o critério principal na avaliação dos métodos.

Utilizando este critério para ranquear as abordagens de aprendizado de acordo com seus respectivos melhores resultados, podemos afirmar que o melhor desempenho foi obtido pelo KNN como classificador de um problema multi-classe seguido por uma combinação de métodos unários, conforme exibido na tabela \ref{tab:resultadosFinais}.

\begin{table}[h]
\centering
\begin{tabulary}{\linewidth}{|L|R|R|R|}
\hline
\textbf{Abordagem}  & \textbf{Precisão} & \textbf{Revocação} & \textbf{F1} \\ \hline
Classificadores multi-classe (KNN)   & 0.889 & 0.727 & 0.800 \\ \hline
Conjunto de classificadores unários  & 0.850 & 0.730 & 0.785 \\ \hline
Classificadores binários (KNN)       & 1.000 & 0.615 & 0.762 \\ \hline
Classificadores unários (OC-SVM)     & 0.842 & 0.615 & 0.711 \\ \hline
\end{tabulary}
\caption{Comparação do desempenho das abordagens de aprendizagem de máquina para a classe de elementos antrópicos, ordenados pela medida F1}
\label{tab:resultadosFinais}
\end{table}


Considerando que existem inúmeros exemplos de elementos antrópicos possíveis em imagens da categoria utilizada neste trabalho e que a base de treinamento possui quantidade e variedade limitadas destes mesmos elementos, é razoável supor que uma abordagem que produza um modelo capaz de reconhecer elementos naturais e não pressupõe características específicas para elementos antrópicos, possa ter melhor desempenho em aplicações reais.

Considerando esse cenário, é válido afirmar que o conjunto de classificadores unários é a abordagem mais promissora dentre as avaliadas para resolver o problema da detecção de elementos antrópicos em imagens aéreas da floresta amazônica. Mesmo que o algoritmo KNN em uma abordagem multi-classe tenha se saído sensivelmente melhor nos experimentos deste trabalho, é esperado que esta modelo de aprendizado obtenha pior desempenho em conjuntos de imagens com novos tipos de elementos antrópicos.

Com uma precisão superior a 85\% e baixa taxa de falsos positivos para a classe de elementos antrópicos, é possível considerar que a abordagem que utiliza conjunto de classificadores unários atinge o objetivo geral deste trabalho: realizar a identificação automática de elementos antrópicos com elevadas taxas de acerto.

Todos os conjuntos de dados, implementações dos algoritmos, ferramentas de apoio e avaliação dos resultados deste trabalho estão publicamente disponíveis\footnote{https://github.com/luizcavalcanti/ForestClassifier} e podem ser verificados, reutilizados e reproduzidos por trabalhos futuros na mesma temática ou em áreas relacionadas.

\section{Trabalhos futuros}

Para o desenvolvimento mais aprofundado de uma solução para a problemática proposta, é preciso continuar explorando o uso de conjuntos de classificadores unários, em especial métodos de \textit{ensemble} de classificadores, com a intenção de aperfeiçoar o método de determinação da classificação final deste conjunto de modelos de aprendizado.

Além disso, é importante que outros algoritmos de classificação unária sejam experimentados, visto que apenas dois, OC-SVM e REPTree, foram utilizados nos experimentos. Testar outros algoritmos pode trazer ganhos na taxa de aprendizado e generalização do modelo gerado.

Também se faz necessária a ampliação do conjunto de imagens disponível para o problema. Embora a base de dados utilizada neste trabalho conte com uma boa quantidade de imagens aéreas da região de floresta amazônica, é preciso incrementar o número de amostras com presença de elementos antrópicos, reduzindo o desequilíbrio na distrbuição das classes. Se possível, novos tipos de elementos antrópicos precisam ser inseridos.