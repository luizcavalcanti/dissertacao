\chapter{Resultados e conclusão}\label{cap:conclusao}

Neste trabalho foi apresentado um estudo comparativo entre diversas abordagens de aprendizagem supervisionada para solucionar o problema da detecção de elementos antrópicos em imagens aéreas da floresta amazônica.

Antes que os experimentos fossem concluídos, foi necessário confeccionar uma base de dados de segmentações manuais, realizada por diversos indivíduos. Este conjunto de dados foi essencial para a avaliação não apenas dos métodos de segmentação considerados estado-da-arte aplicados ao problema, mas também para a avaliação da consistência da segmentação humana na coleção de imagens em questão.

Dentre as técnicas de segmentação analisadas, o método SRM, que realiza o agrupamento de pixels em uma região de acordo com a similaridade da intensidade e cor dos pixels de regiões vizinhas, obteve erros de consistência global e local inferiores aos demais. É importante notar também que o algoritmo MSEG, que se utiliza de informações de cor e textura, obteve desempenho aproximado ao do SRM, mas com o tempo de execução cerca de quatorze vezes menor. Isto torna o algoritmo MSEG um forte candidato a substituir a técnica SRM em uma aplicação prática, em que o tempo de processamento da segmentação seja um fator de impacto.

Um subproduto do experimento que pretendeu analisar as técnicas de segmentação é uma base de cerca de 10.000 segmentos, devidamente identificados e rotulados. Este é um importante objetivo deste trabalho, pois, junto com as ferramentas de software desenvolvidas para facilitar o processo de classificação das amostras, esta base de dados possibilita que trabalhos futuros sejam realizados tanto na melhoria dos resultados obtidos quanto na reaplicação em outros temas.

Para a classificação dos segmentos e, por consequência, a detecção de elementos anômalos nas imagens, quatro abordagens de modelagem do problema de aprendizagem de máquina foram avaliadas.

A primeira delas utiliza classificadores multi-classe em uma base com rótulos de cinco classes (floresta, vegetação rasteira, água, terra e elementos antrópicos). Neste experimento o algoritmo \textit{K-Nearest Neighbor} (KNN) com $k=3$ obteve os melhores resultados de todo o trabalho. Apesar dos bons resultados, técnicas de aprendizado preguiçoso (\textit{lazy learning}) como o KNN não criam um modelo de aprendizado sucinto, necessitando de boa parte das amostras de treinamento para realizar a classificação de novas amostras, o que pode ser problemático em um ambiente real, com limitações de tempo de processamento e memória disponível.

Adicionalmente, um problema da abordagem com classificadores multi-classe é a pouca generalização do aprendizado sobre elementos antrópicos, visto que se trata de uma classe pouco previsível. Como os elementos antrópicos fazem parte do modelo de treinamento, novas amostras de objetos antrópicos cujas características se distanciem das amostras de elementos antrópicos da base de treinamento tendem a reduzir a precisão do modelo.

A segunda abordagem analisada, que utiliza classificadores binários, obteve desempenho acima do esperado. A expectativa era de que os modelos gerados por esta abordagem tivessem um baixo desempenho na classificação da classe de elementos antrópicos, uma vez que esta correspondia a apenas 0.3\% da base de dados. Esperava-se também que por, conta da re-classificação das amostras de segmentos que reagrupou as classes de floresta, vegetação rasteira, água e terra em uma única classe, o modelo gerado fosse pouco descritivo, hipótese que não pôde ser comprovada.

Neste experimento o algoritmo KNN também obteve o melhor desempenho para a classe de elementos antrópicos, desta vez se beneficiando da seleção de características feita através da técnica CFS, que utiliza a correlação entre os atributos do problema para reduzir a dimensão do vetor de características. Os resultados indicam que a seleção de atributos contribuiu significativamente para a simplificação do modelo gerado, tornando-o mais generalista e mais apto a descrever a classe de elementos naturais, resultado inverso ao experimento com classificadores multi-classe.

\todo[inline]{Falar do experimento de unários}

\todo[inline]{Falar do experimento de ensemble}

Como métrica para avaliação das abordagens de aprendizado ficou definido que a medida $F1$, que leva em consideração um balanço entre precisão e revocação, seria utilizada para julgar o desempenho dos algoritmos. Por conta da importância da classe, bem como forma de compensar o desequilíbrio na distribuição das classes do problema na base de dados de treinamento, a medida $F1$ da classe de elementos antrópicos foi o critério principal na avaliação dos métodos.

Utilizando este critério para ranquear as abordagens de aprendizado de acordo com seus respectivos melhores resultados, podemos afirmar que o melhor desempenho foi obtido pelo KNN como classificador de um problema multi-classe seguido por uma combinação de métodos unários, conforme exibido na tabela \ref{tab:resultadosFinais}.

\begin{table}[h]
\centering
\begin{tabulary}{\linewidth}{|L|R|R|R|}
\hline
\textbf{Abordagem}  & \textbf{Precisão} & \textbf{Revocação} & \textbf{F1} \\ \hline
Multi-classe (KNN)  & 0.889 & 0.727 & 0.800 \\ \hline
Conjunto de unários & 0.850 & 0.730 & 0.785 \\ \hline
Binária (KNN)       & 1.000 & 0.615 & 0.762 \\ \hline
Unária (OC-SVM)     & 0.842 & 0.615 & 0.711 \\ \hline
\end{tabulary}
\caption{Comparação de abordagens de aprendizagem de máquina para a classe de elementos antrópicos, ordenados pela medida F1}
\label{tab:resultadosFinais}
\end{table}

\todo[inline]{Falar do quão promissor foi a abordagem de ensemble de classificadores unários}

\todo[inline]{Falar que tudo é público e está disponível no github}

\section{Trabalhos futuros}

\todo[inline]{Continuar explorando ensembles}
\todo[inline]{Explorar novos métodos de apuração e novos algoritmos de oneclass}
