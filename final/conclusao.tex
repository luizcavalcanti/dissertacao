\chapter{Conclusão e Trabalhos Futuros}\label{cap:conclusao}

Este trabalho teve como objetivo propor uma abordagem eficiente, utilizando processamento digital de imagens  e aprendizagem de máquina, para o problema de detecção automática de elementos antrópicos em imagens aéreas da floresta amazônica. A intenção era demonstrar que a identificação automática destes elementos pode trazer benefícios, em especial agilidade, ao processo que hoje é completamente dependente de agentes humanos, e produzir elevadas taxas de acerto.

Em um primeiro momento, uma base de dados foi criada, rotulada e disponibilizada publicamente para que experimentos pudessem ser executados, e também que pudesse servir de referência para futuros estudos nesta área de pesquisa e em áreas relacionadas.

Posteriormente, investigamos e definimos métodos de segmentação de imagens adequados à este tipo de imagem e problema. O algoritmo SRM obteve o melhor desempenho e foi indicado para esta etapa da solução. É importante ressaltar que outros algoritmos de segmentação como o MSEG também obtiveram resultados promissores, e não devem ser descartados em problemáticas similares.

Em seguida, características foram extraídas das imagens, para que um vetor de atributos pudesse ser usado na etapa de classificação dos segmentos. Elementos de baixo nível como cor e textura foram mesclados à características de alto nível como textura, para representar as amostras de segmentos de imagens aéreas. A técnica de seleção de atributos CFS foi utilizada para escolher um subconjunto de atributos a ser utilizado em experimentos específicos, muitas vezes com resultados melhores ao conjunto completo de atributos.

Por último, quatro abordagens de aprendizagem de máquina supervisionada foram avaliadas: classificação multi-classe, binária, unária e conjuntos de classificadores unários. Esta última abordagem obteve o melhor desempenho, visto que teve boa taxa de aprendizado geral e elevada taxa de detecção da classe de elementos antrópicos, assim como tem um bom potencial de generalização da capacidade de detectar novos elementos antrópicos, não presentes na base utilizada neste trabalho.

Todos os conjuntos de dados, implementações dos algoritmos, ferramentas de apoio e avaliação dos resultados deste trabalho estão publicamente disponíveis\footnote{https://github.com/luizcavalcanti/ForestClassifier} e podem ser verificados, reutilizados e reproduzidos por trabalhos futuros na mesma temática ou em áreas relacionadas.

\section{Trabalhos futuros}

Para o desenvolvimento mais aprofundado de uma solução para a problemática proposta, é preciso continuar explorando o uso de conjuntos de classificadores unários, em especial métodos de \textit{ensemble} de classificadores, com a intenção de aperfeiçoar o método de determinação da classificação final deste conjunto de modelos de aprendizado.

Além disso, é importante que outros algoritmos de classificação unária sejam experimentados, visto que apenas dois, OC-SVM e REPTree, foram utilizados nos experimentos. Testar outros algoritmos pode trazer ganhos na taxa de aprendizado e generalização do modelo gerado.

Também se faz necessária a ampliação do conjunto de imagens disponível para o problema. Embora a base de dados utilizada neste trabalho conte com uma boa quantidade de imagens aéreas da região de floresta amazônica, é preciso incrementar o número de amostras com presença de elementos antrópicos, reduzindo o desequilíbrio na distribuição das classes. Se possível, novos tipos de elementos antrópicos precisam ser inseridos.