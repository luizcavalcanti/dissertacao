\chapter{Conclusão}\label{cap:conclusao}

\todo[inline]{Trabalho em progresso}

%Neste trabalho foi apresentado uma solução para detecção de elementos antrópicos em imagens aéreas da floresta amazônica, utilizando técnicas de processamento digital de imagens e aprendizado de máquina.

%Começamos este trabalho preparando uma base de imagens e um conjunto de ferramentas auxiliares...


\textbf{Pontos importantes:}
\begin{itemize}
	\item Falar da base de segmentos rotulada disponibilizada
	\item Falar do benchmark de segmentação e das ferramentas desenvolvidas para segmentação manual
	\item Falar dos bons resultados do KNN mas também das desvantagens de não ter um modelo (método não-paramétrico, etc)
	\item Falar do quão promissor foi a abordagem de ensemble de classificadores unários
\end{itemize}

\section{Trabalhos futuros}

\textbf{Pontos importantes:}
\begin{itemize}
	\item Continuar explorando ensembles
	\item Explorar novos métodos de apuração e novos algoritmos de oneclass
\end{itemize}