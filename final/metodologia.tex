\chapter{Metodologia}\label{cap:metodologia}

Neste capítulo é descrita a sequência de etapas que serão realizadas neste trabalho para que os objetivos de pesquisa sejam alcançados.

Em imagens aéreas como as utilizadas neste trabalho, é comum que os elementos que indicam presença humana sejam relativamente grandes (pista de pouso, estradas, clareiras, etc.), podendo ser definidas como uma região durante a segmentação da imagem.

Classificar pixels tende a ser custoso, visto que mesmo uma pequena imagem provê dezenas de milhares deles, que devem ter suas características extraídas e providas ao modelo de aprendizado, para que possam ser classificados. Portanto, utilizar técnicas de segmentação de imagem para agrupar os pixels espacialmente e caracteristicamente relacionados em uma única amostra não só possibilita uma execução mais rápida da solução computacional, como também torna o resultado final menos ruidoso.

Por este motivo, a arquitetura para a solução proposta prevê uma etapa de segmentação das imagens, seguida de uma etapa de classificação, responsável pela determinação do tipo de cada região encontrada na segmentação. A arquitetura geral da solução pode ser vista na figura \ref{fig:metDiagramaGeral}.

\begin{figure}[h]
    \includegraphics[width=\textwidth]{imgs/arquitetura_geral}
    \caption{Arquitetura geral da solução a ser desenvolvida para detecção de elementos antrópicos em imagens aéreas da floresta amazônica}
    \label{fig:metDiagramaGeral}
\end{figure}

A ideia geral é que imagens aéreas de regiões florestais da Amazônia legal sirvam de entrada para o problema. Estas mesmas imagens serão particionadas em regiões por um segmentador. Em seguida, um extrator será utilizado para criar o vetor de características de cada segmento provido pela etapa anterior. Por fim um classificador, composto por um ou mais algoritmos de classificação,  será utilizado para rotular as regiões com a finalidade de encontrar as regiões de elementos antrópicos. A saída do sistema pode, então, ser composta pelas regiões segmentadas e suas classificações finais.

Embora esta seja uma arquitetura simples, bastante difundida na literatura, nuances para o problema apresentado neste trabalho devem ser levadas em conta. Para chegarmos à conclusão de quais segmentadores, extratores e classificadores devem ser utilizados, são necessárias várias etapas de experimentação, validação de resultados e análises.

\section{Entrada}

Uma base de dados com imagens aéreas de floresta tropical precisa ser formada. Todas as imagens precisam ter as mesmas dimensões, condições visuais semelhantes e preferencialmente pertencerem à região de floresta amazônica, excluindo imagens de cidades e povoados da região.

Como serão processada por uma etapa de segmentação, as imagens não sofrerão nenhum tipo de filtragem, ficando estas a cargo dos segmentadores a serem avaliados. A base de imagens gerada nesta etapa é um dos objetivos específicos deste trabalho.

\section{Segmentador}\label{sec:metSegmentador}

Para encontrarmos o segmentador ideal, os métodos de segmentação de imagens considerados estado-da-arte serão aplicados à uma parte da base de imagens aéreas da floresta amazônica. Esta porção da base de imagens também será manualmente segmentada por seres humanos e servirá de base de comparação para a segmentação realizada pelos métodos experimentados. Ainda é preciso aferir a consistência da segmentação manual realizada por seres humanos. Métricas de erros de consistência como o \textit{Local Consistency Error} (LCE) e \textit{Global Consistency Error} (GCE) devem ser aplicados.

Esta etapa deve determinar o método de segmentação com melhores resultados, a ser utilizado na solução descrita pelo trabalho. A precisão e o tempo de execução devem ser utilizados para a avaliação do desempenho dos algoritmos testados. Em seguida, é preciso extrair as características dos segmentos gerados.

\section{Extrator}\label{sec:metExtrator}

Os algoritmos de classificação lidam com variáveis numéricas inteiras, de ponto flutuante e em alguns casos, dados textuais. No entanto, para classificar imagens, ou no caso deste trabalho, segmentos de imagens, é preciso que um conjunto de características seja extraído dos pixels destas imagens ou regiões e disponibilizados para o modelo de classificação.

Nesta etapa serão testadas diversas características disponíveis na literatura de processamento digital de imagens, especialmente informações sobre cor, intensidade, textura e morfologia destas amostras. Também deverão ser testadas técnicas para redução da dimensionalidade do vetor de amostras, tais quais técnicas que se utilizam de correlação e ganho de informação.

Como requisito da composição da base de dados de segmentos, bem como para avaliação e otimização do vetor de características, todas as amostras geradas pela etapa de segmentação devem ser rotuladas manualmente e contar com a avaliação de especialistas na inspeção deste tipo de imagem. Com a base devidamente criada e rotulada, experimentos para determinar os classificadores mais adequados podem ser feitos.

\section{Classificador}\label{sec:metClassificador}

Nesta etapa, quatro estratégias de classificação presentes na literatura serão abordadas. Elas não demandam alterações na arquitetura geral da solução, mas mudam a forma como a base é rotulada e organizada. As métricas de avaliação do aprendizado precisam ser ponderadas para cada estratégia.

É importante destacar que a classe de interesse, elementos antrópicos, permanece a mesma em todas as abordagens, e sua inspeção mais cuidadosa é vital na avaliação de todos os métodos utilizados.

\subsection{Classificador multi-classe}

Nesta estratégia, a base de segmentos gerada será rotulada entre cinco classes possíveis: floresta, vegetação rasteira, água, terra e elemento antrópico. Diversos algoritmos que possibilitam clasificação multi-classe serão utilizados, tendo em mente que é importante testar métodos simbólicos, bayesianos, não-paramétricos, máquina de vetores de suporte, entre outros.

Todos os métodos devem ser testados com a base completa de segmentos, mas também com a base pré-processada por um seletor de características, que será responsável pela redução do vetor de características do problema. Este detalhe do experimento servirá para determinar se a seleção de atributos nesta abordagem reduz a complexidade dos modelos gerados e o ruído na base de dados, possibilitando melhor taxa de aprendizagem.

Cada algoritmo deve ser responsável pela classificação de toda a base. Ao fim, métricas de aprendizado como acurácia, precisão e revocação serão utilizadas para comparar os métodos entre si.

\subsection{Classificador binário}

Neste experimento, a base de segmentos gerada será rotulada entre duas classes: elementos naturais e elementos antrópicos. A classe de elementos naturais agrupa o que originalmente seriam as classes de floresta, vegetação rasteira, água e terra.

Os classificadores utilizados serão os mesmos do experimento com classificadores multi-classe, visto que não há grande diferença metodológica nas duas abordagens. Todos os métodos também serão testados com seletores de características, e o impacto desta seleção também será avaliado.

Ao fim do experimento, as métricas propostas para o problema de aprendizado serão colhidas para cada método testado e utilizados para comparar os métodos entre si, bem como o impacto geral da seleção de atributos para esta abordagem.

\subsection{Classificador unário}

Assim como no experimento de classificadores binários, a base de dados de segmentos será dividida entre elementos naturais e elementos antrópicos, utilizando os mesmos critérios. À primeira vista, esta estratégia de aprendizado parece muito similar aos classificadores binários, mas a diferença está inicialmente nos algoritmos utilizados.

Como o aprendizado unário utiliza o conceito de classe majoritária e classe anômala (\textit{outlier}), é preciso definir quem é o quê no problema proposto. Por uma simples questão de frequência estatística, fica definido que a classe de elementos naturais é a classe majoritária, enquanto a classe de elementos antrópicos é considerada a classe anômala.

\todo[inline]{concluir seção}

\subsection{Ensemble de classificadores unários}

\todo[inline]{concluir seção}

Elemento natural (floresta, vegetação rasteira, água, terra) e elemento antrópico

\section{Saída}

O resultado dos experimentos deve sustentar a conclusão sobre quais métodos de segmentação e classificação são mais indicados para a solução do problema. O próximo capítulo descreve os experimentos realizados e discute os resultados encontrados.