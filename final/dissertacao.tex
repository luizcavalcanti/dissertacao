\documentclass[
    12pt,
    a4paper,
    english,
    brazil,
    sumario=tradicional]{abntex2}
\usepackage{lmodern}
\usepackage[T1]{fontenc}
\usepackage[utf8]{inputenc}
\usepackage{lastpage}
\usepackage{indentfirst}
\usepackage{color}
\usepackage{graphicx}
\usepackage{caption}
\usepackage{subcaption}
\usepackage{microtype}
\usepackage[brazil]{babel}
\usepackage[brazilian,hyperpageref]{backref}
\usepackage[alf]{abntex2cite}
\usepackage{multirow}
\usepackage{gensymb}
\usepackage[table]{xcolor}
\usepackage{tabulary}
\usepackage{pdfpages}

%%%% Configurações %%%%
\renewcommand{\backrefpagesname}{Citado na(s) página(s):~}
\renewcommand{\backref}{}
\renewcommand*{\backrefalt}[4]{
	\ifcase #1 %
		Nenhuma citação no texto.%
	\or
		Citado na página #2.%
	\else
		Citado #1 vezes nas páginas #2.%
	\fi}%

\setlength{\parindent}{1.3cm}
\setlength{\parskip}{0.2cm}

%%%% Constantes de autoria %%%%

\titulo{Uma solução de aprendizagem de máquina para detecção de elementos antrópicos em imagens aéreas da floresta amazônica}
\autor{Luiz Carlos A. M. Cavalcanti}
\local{Manaus}
\data{2015}
\orientador[Orientadora:]{Prof\textsuperscript{a}. Dra. Eulanda Miranda dos Santos}
\instituicao{
    PODER EXECUTIVO
    \par
    MINISTÉRIO DA EDUCAÇÃO
    \par
    UNIVERSIDADE FEDERAL DO AMAZONAS
    \par
    INSTITUTO DE COMPUTAÇÃO
    \par
    PROGRAMA DE PÓS-GRADUAÇÃO EM INFORMÁTICA
}

\preambulo{Proposta de dissertação apresentada ao Programa de Pós-graduação em Informática da Universidade Federal do Amazonas como requisito parcial para obtenção do grau de Mestre em Informática. 
Área de concentração: Visão Computacional e Robótica}

\tipotrabalho{Dissertação (Mestrado)}

%%%% Documento %%%%
\begin{document}

\imprimircapa
\imprimirfolhaderosto

\pagenumbering{roman}

\begin{folhadeaprovacao}
    \begin{center}
        {\ABNTEXchapterfont\large\imprimirautor}
        \vspace*{\fill}\vspace*{\fill}
        \begin{center}
            \ABNTEXchapterfont\bfseries\Large\imprimirtitulo
        \end{center}
        \vspace*{\fill}
    \hspace{.45\textwidth}
    \begin{minipage}{.5\textwidth}
        \imprimirpreambulo
    \end{minipage}%
    \vspace*{\fill}
    \end{center}

    Trabalho aprovado. \imprimirlocal, \today:
    \assinatura{\textbf{\imprimirorientador} \\ Orientadora}
    \assinatura{\textbf{Professor} \\ Convidado 1}
    \assinatura{\textbf{Professor} \\ Convidado 2}

    \begin{center}
        \vspace*{0.5cm}
        {\large\imprimirlocal}
        \par
        {\large\imprimirdata}
        \vspace*{1cm}
    \end{center}
\end{folhadeaprovacao}

\begin{dedicatoria}
	\vspace*{\fill}
	A dedicatória...
	\vspace*{\fill}
\end{dedicatoria}

\begin{agradecimentos}
   Os agradecimentos...
\end{agradecimentos}

\begin{epigrafe}
	\vspace*{\fill}
	\begin{flushright}
		A epígrafe...
	\end{flushright}
\end{epigrafe}

\tableofcontents*
\clearpage

\listoffigures
\clearpage

\listoftables
\clearpage

\begin{siglas}
	\item[BSD500] Berkeley Segmentation Data Set and Benchmarks 500
	\item[CENSIPAM] Centro Gestor e Operacional do Sistema de Proteção da Amazônia
	\item[CFS] Correlation-based Feature Subset Selection
	\item[IBGE] Instituto Brasileiro de Geografia e Estatística
	\item[INPE] Instituto Nacional de Pesquisas Espaciais
	\item[FSEG] Factorisation-Based Segmentation
	\item[GCE] Global Consistency Error
	\item[JPEG] Joint Photographic Experts Group
	\item[KNN] K-Nearest Neighbours
	\item[LBP-HF] Local Binary Pattern Histogram Fourier
	\item[LCE] Local Consistency Error
	\item[LIDAR] Light Detection and Ranging
	\item[OCNN] One-Class Nearest Neighbours
	\item[OC-SVM] One-Class Support Vector Machines
	\item[PDI] Processamento Digital de Imagens
	\item[RADAR] Radio Detection and Ranging
	\item[ROC] Receiver Operating Characteristic
	\item[SRM] Statistical Region Merging
	\item[SVM] Support Vector Machines
	\item[UAV] Unmanned Aerial Vehicle
	\item[VANT] Veículo Aéreo Não-Tripulado
\end{siglas}


\pagenumbering{arabic}

\begin{resumo}
    Durante o patrulhamento de crimes ambientais, o tempo de resposta é um componente muito importante no sucesso das missões. Geralmente as infrações ocorrem em lugares ermos e de difícil acesso, características que dificultam tanto o patrulhamento quanto a ação de agentes de preservação ambiental. Para aumentar a taxa de sucesso das abordagens e reduzir o risco de vidas humanas, veículos aéreos não-tripulados (VANTs) podem ser usados para cobrir grandes áreas de floresta em pouco tempo, sem que sejam percebidos por infratores, permitindo que os órgãos de patrulhamento dessas áreas possam planejar e agir com mais eficiência na repressão a esses crimes. O novo problema gerado por essa abordagem é a enorme quantidade de dados gerada durante essas missões, que muitas vezes compreendem horas de vídeo. A inspeção manual de todo esse material em busca de elementos antrópicos é muito cansativa e propensa a erros humanos. Esta pesquisa de mestrado propõe realizar a detecção automática de objetos anômalos ao ambiente natural, tais como acampamentos, pistas, estradas e embarcações em imagens aéreas da floresta amazônica. O objetivo final deste trabalho é utilizar técnicas computacionais para reduzir o tempo de análise e a quantidade de dados que precisam ser avaliados por especialistas humanos.

    \vspace{\onelineskip}
    \noindent
    \textbf{Palavras-chaves}: aprendizado de máquina, segmentação, classificação, imagens aéreas
\end{resumo}

\begin{resumo}[Abstract]
    During environmental crimes patrolling, the response time is a very important component in the success of the missions. Generally, infractions occur in remote and hard-access places, characteristics that hinder both the patrolling as well the action of environmental protection agents. To increase the approaches' success rate and reduce the risk of human lives, unmanned aerial vehicles (UAVs) can be used to cover large areas of forest in a short time without being perceived by offenders, allowing the patrolling organs responsible for these areas to plan and act more efficiently in the repression of such crimes. The new problem generated by this approach is the huge amount of data generated during these missions, which often includes hours of video. The manual inspection of all this material in searching for anthropic elements is very tiring and prone to human error. This master's research proposes to perform automatic detection of anomalous objects in natural environment, such as campgrounds, trails, roads and boats in aerial images of the Amazon forest. The goal of this work is to apply computational techniques to reduce the time of analysis and the amount of data that needs to be assessed by human experts.
    
    \vspace{\onelineskip}
    \noindent
    \textbf{Keywords}: machine learning, segmentation, classification, aerial images
\end{resumo}


\textual

\chapter{Introdução}

Os órgãos de segurança e de patrulhamento ambiental lidam diariamente com missões de alto risco. Essas missões têm grande custo de operação e representam um perigo significativo para os agentes humanos envolvidos. Especificamente na região da floresta amazônica, a área a ser patrulhada possui uma dimensão ímpar: 11 mil quilômetros de fronteiras e 22 mil quilômetros de rios, compondo uma área total de 5.217.423 quilômetros quadrados do que é conhecido como Amazônia Legal, o que corresponde à aproximadamente 61\% do território brasileiro \cite{ibge}.

Com o intuito de minimizar os custos, aumentar a eficiência e mitigar os riscos humanos de diversos tipos de atividades, são empregados Veículos Aéreos Não-Tripulados (VANTs). Trata-se de veículos autônomos ou controlados remotamente, que têm como objetivo a agilidade e discrição no reconhecimento de áreas isoladas ou hostis, antes que uma equipe de agentes humanos precise intervir. Essa abordagem permite que as eventuais ameaças sejam previamente identificadas, e o comando tático possa tomar as melhores decisões sobre quando e como agir na região em questão.

Os VANTs têm sido utilizados em uma miríade de missões de diferentes naturezas, com destaque para o combate a crimes em áreas de fronteira. Na região da Floresta Amazônica, os veículos podem ser utilizados no combate ao tráfico de animais, à extração ilegal de madeira e outros crimes ambientais \cite{silva:2013}.

Por acontecerem em regiões isoladas e de difícil acesso, os responsáveis por essas infrações costumam escapar antes que os agentes de segurança possam agir. A aproximação de humanos alerta os infratores, pois os veículos (barcos, helicópteros e veículos terrestres utilitários) usados na abordagem produzem sons bastante perceptíveis nesses ambientes. Por esses motivos, os VANTs representam uma boa alternativa para possibilitar flagrantes e o planejamento tático dessas missões.

Equipados com câmeras de diversos tipos e alcances, esses veículos são capazes de gerar muitas horas de vídeo a cada missão, tornando a análise simultânea ou posterior do material gerado um grande esforço para as pessoas responsáveis pelo planejamento tático, presentes em salas de controle para onde os vídeos são transferidos ao vivo, ou analisados posteriormente.

\section{Motivação}

A quantidade maciça de dados gerados pelos VANTs leva à necessidade de desenvolvimento de procedimentos capazes de avaliar essa grande quantidade de material e identificar os prováveis pontos de interesse. Um sistema capaz de fazer tal análise seria importante para possibilitar a rápida avaliação do material gerado por uma operação, ou mesmo para que uma quantidade menor de profissionais seja necessária para o acompanhamento em tempo real de diversos VANTs em uma mesma missão. Isso cria uma demanda para a filtragem e classificação desses dados em informações ou alertas relevantes aos agentes humanos que supervisionam esses equipamentos.

Ao provar a eficiência de uma implementação automatizada dessa classificação, um significativo aumento pode ser alcançado na qualidade e agilidade da detecção de elementos antrópicos em ambientes silvestres, bem como na redução da margem de erro dos agentes humanos envolvidos. Através de processamento de imagens digitais e reconhecimento automático de padrões, é possível avaliar o material com maior rapidez, e sinalizar os pontos de possível interesse dos profissionais que acompanham as missões.

Baseando-se em elementos antrópicos encontrados nas imagens aéreas de regiões silvestres, áreas com relativa previsibilidade de paisagem, é possível buscar padrões e reconhecer ameaças ou interesses que possam passar despercebidos por olhos humanos, especialmente em um material com extensão de horas. Tais elementos antrópicos incluem acampamentos de caçadores, pistas clandestinas, estradas, embarcações, ou seja, indícios de presença humana em áreas de selva.

Diante desse contexto, este projeto propõe o uso de técnicas de processamento digital de imagens e de reconhecimento de padrões para detectar elementos antrópicos em imagens obtidas por veículos aéreos, de uma região da Floresta Amazônica. A detecção de elementos antrópicos refere-se à identificação de objetos estranhos ao padrão normal da floresta, de provável origem humana.

\section{Objetivos}

\subsection{Geral}

Este trabalho tem como objetivo propor uma solução computacional que envolva Processamento Digital de Imagens (PDI) e aprendizagem de máquina para o problema de detecção de elementos antrópicos em imagens aéreas da floresta amazônica, com a intenção de demonstrar que a identificação automática destes elementos pode trazer benefícios, em especial agilidade, ao processo que hoje é completamente dependente de agentes humanos, e produzir elevadas taxas de acerto.

\subsection{Específicos}

\begin{itemize}
    \item Organizar uma base de dados de imagens devidamente rotuladas que sirva de referência para futuros estudos desta problemática.
    \item Investigar e definir métodos para extração e seleção de características mais adequadas ao problema em questão.
    \item Realizar experimentos e apontar o melhor conjunto de técnicas para a detecção de elementos antrópicos em imagens aéreas da floresta amazônica.
\end{itemize}

\section{Organização do Documento}

O restante deste documento está organizado da seguinte forma: no capítulo \ref{cap:fundamentacao} discorremos sobre a fundamentação teórica necessária para o entendimento do trabalho proposto. No capítulo \ref{cap:trabalhos} os trabalhos relacionados são apresentados, com ordem e critérios descritos neste mesmo capítulo. Uma comparação das referências também é feita. No capítulo \ref{cap:metodologia} é descrita a metodologia que será empregada no desenvolvimento deste trabalho de pesquisa. No capítulo \ref{cap:experimentos} são apresentadas a base de dados, sua preparação e a descrição dos experimentos realizados, com os resultados alcançados. Por fim, no capítulo \ref{cap:conclusao} uma conclusão para o trabalho é apresentada.


\chapter{Fundamentação Teórica}

Neste capítulo são discutidos os conceitos básicos ao entendimento deste trabalho, abrangendo processamento digital de imagens, segmentação de imagens, aprendizagem de máquina, extração de características e avaliação de desempenho dos métodos utilizados.

\section{Processamento Digital de Imagens}

Uma imagem pode ser definida por uma função bi-dimensional, $f(x,y)$, onde $x$ e $y$ são coordenadas espaciais de um plano, e a amplitude de $f$ para um par de coordenadas $(x,y)$ é chamada de intensidade da imagem naquele ponto. Quando $x$, $y$ e a amplitude de valores de $f$ são todas quantidades finitas e discretas, podemos chamar a imagem de imagem  digital. A linha de pesquisa de processamento digital de imagens se refere ao processamento digital dessas imagens através de um computador digital \cite{gonzalez:2002}.

Uma imagem digital é composta por um número finito de elementos, cada um com um valor ($f$) e localização ($x$ e $y$) particulares. Estes elementos são chamados de elementos da imagem ou \textit{pixels}, do inglês \textit{picture elements}.

Ainda de acordo com \citeonline{gonzalez:2002}, o interesse em processamento digital de imagens advém de duas principais áreas de aplicação: melhoramento de informações pictóricas para interpretação humana; e processamento de imagens para armazenamento e transmissão de dados, e representação para percepção de máquinas autônomas.

\subsection{Segmentação de Imagens}

O processo de segmentação subdivide uma imagem em suas várias regiões ou objetos. O nível em que a subdivisão é feita depende do problema a ser resolvido, ou seja, a segmentação deve parar quando os objetos de interesse de uma aplicação forem isolados. Por exemplo, na inspeção automática de uma linha de montagem de produtos eletrônicos, o interesse reside em analisar imagens dos produtos com o objetivo de determinar a presença ou ausência de anomalias específicas, como componentes faltando ou conexões quebradas. Não há sentido em continuar segmentando além do nível de detalhes necessário para a identificação destes elementos.

Segmentação de imagens é um dos problemas mais difíceis em processamento digital de imagens \cite{gonzalez:2002}. Algoritmos de segmentação geralmente se baseiam em uma das duas propriedades básicas de valores de intensidade dos pixels: descontinuidade e similaridade. Na primeira propriedade, a abordagem é particionar a imagem baseando-se em mudanças abruptas na intensidade dos pixels, como as bordas. A segunda categoria se baseia no particionamento de uma imagem em regiões que são similares de acordo com um critério em particular, que pode ser coloração, textura, entre outros.

Neste trabalho, conforme será descrito no capítulo \ref{cap:metodologia}, testamos uma série de atributos para determinar a melhor forma de segmentar as regiões das imagens aéreas da floresta amazônica, como uma das etapas da pesquisa de detecção de anomalias.

Muitas vezes, técnicas de segmentação de imagens não são suficientes para isolar ou detectar componentes específicos na coleção de imagens. Trabalhos recentes têm utilizado, com frequência, algoritmos de aprendizagem de máquina para resolver este problema.


\section{Aprendizagem de máquina e reconhecimento de padrões}

Conforme \citeonline{alpaydin:2010}, aprendizagem de máquina é uma área da inteligência artificial que estuda métodos computacionais, a fim de obter um determinado conhecimento específico através de experiências. A aplicação prática de aprendizado de máquina inclui o processamento de linguagem natural, diagnósticos médicos, detecção de intrusos, entre outros. Um sistema de aprendizado tem a função de analisar informações e generalizá-las, para a extração de novos conhecimentos.

Segundo \citeonline{russell:2010}, os tipos de aprendizagem podem ser classificados de acordo com o tipo de \textit{feedback} que recebem do ambiente:

\begin{itemize}
    \item Aprendizagem não-supervisionada: o agente aprende padrões na entrada, embora não seja fornecido nenhum \textit{feedback} explícito. A tarefa mais comum de aprendizagem não-supervisionada é o agrupamento, ou seja, a detecção de grupos de exemplos de entrada potencialmente úteis.
    \item Aprendizagem por reforço: também conhecida como aprendizagem semi-supervisionada. O agente aprende a partir de uma série de reforços - recompensas ou punições.
    \item Aprendizagem supervisionada: o agente observa alguns exemplos de pares de entrada e saída, e aprende uma função que faz o mapeamento da entrada para a saída.
\end{itemize}

Os problemas de aprendizagem podem ainda ser divididos de acordo com o tipo de saída que demandam:

\begin{itemize}
	\item Problemas de classificação: quando a saída esperada para o problema é uma classe ou categoria, ou seja, um valor discreto;
	\item Problemas de regressão: quando a saída esperada para o problema é um valor numérico, normalmente contínuo.
\end{itemize}

Um problema de classificação, ou seja, um problema em que o objetivo é atribuir corretamente classes discretas (rótulos) aos exemplos de dados, consiste na determinação de regras e posterior classificação desses exemplos. Este conjunto de regras é criado por um classificador, que recebe como entrada um vetor de características e oferece como saída uma classe resultante para a instância que as características descrevem, conforme pode ser visto na figura \ref{fig:classificador}.

\begin{figure}[h!]
  \centering
  \includegraphics[width=0.7\textwidth]{imgs/classificador}
  \caption{Representação de classificador como uma função de bloco}
  \label{fig:classificador}
\end{figure}

Os tipos de classificadores utilizados neste trabalho serão discutidos com mais detalhes na seção \ref{sec:classificacao}.

Para composição do modelo de aprendizagem, uma base de dados de treinamento é utilizada. Esta base deve possuir uma quantidade significativa e com boa representatividade das classes envolvidas no problema. Normalmente se usa uma parte da base de dados de treinamento para validação do modelo de aprendizado (validação cruzada) ou mesmo uma base de dados diferente (base de testes ou validação), para que indicadores de qualidade do modelo possam ser avaliados. A seção \ref{sec:avaliacao} discorre sobre os métodos de avaliação utilizados neste trabalho.

Técnicas de aprendizagem de máquina podem ser utilizadas para encontrar padrões em diversos domínios, inclusive em imagens. É neste ponto que a linha de pesquisa de aprendizagem de máquina, advinda da área de inteligência artificial, se encontra com a linha de pesquisa de reconhecimento de padrões, advinda da área de processamento de sinais. Segundo \citeonline{jain:1989}, o fluxo padrão para soluções de reconhecimento de padrões consiste em três etapas:

\begin{enumerate}
    \item Filtragem e pré-processamento da entrada;
    \item Extração e seleção de características;
    \item Classificação.
\end{enumerate}


\subsection{Filtragem e Pré-processamento}

A etapa de filtragem e pré-processamento é responsável pela escolha e montagem da base de dados que será usada no processo de aprendizagem. A base deve conter uma quantidade significativa de exemplos de todas as classes envolvidas no problema.

Em aprendizado relacionado a imagens, essa etapa é comumente a responsável por normalizar e salientar as características desejadas nas amostras (realce de imagens, filtragem, etc). Exemplos irrelevantes, distorcidos ou repetidos também são eliminados durante a filtragem. O objetivo principal desta etapa é preparar a base de dados para as etapas seguintes.


\subsection{Extração de Características}

A extração de características é feita selecionando os atributos oriundos dos dados (imagens, no trabalho em questão), a fim de encontrar as características úteis para o processo de reconhecimento. Essa etapa é crítica ao sucesso do aprendizado, uma vez que bons algoritmos de aprendizado só obtêm êxito com um bom conjunto de características relevantes ao problema.

Em projetos que envolvem classificação de imagens, uma gama de atributos pode ser extraída, e podem ser descritos pelo nível da informação que representam. Nesta etapa há uma forte contribuição da linha de pesquisa de processamento digital de imagens \cite{gonzalez:2002}, que descreve filtragens, transformações e outras técnicas capazes de extrair informações sobre uma imagem ou pedaços dela.

Segundo \citeonline{nixon:2008}, informações de baixo-nível como bordas, histogramas de intensidade e coloração, são úteis para o reconhecimento de padrões em imagens, assim como características de níveis mais altos, como texturas, transformadas de Hough e extração de regiões conectadas.

O produto desta etapa é a representação de cada exemplo da base de dados em um vetor de características, de forma que possa ser usado por um ou mais classificadores em uma etapa posterior.


\subsection{Classificação}\label{sec:classificacao}

Nesta etapa, todas as amostras de treinamento são classificadas e um modelo de aprendizado é gerado. Posteriormente ao processo de aprendizado, é nesta mesma etapa que as amostras não classificadas receberão uma classe dentre as envolvidas no problema. É neste momento que podemos comparar o desempenho de diferentes algoritmos de aprendizado para o conjunto de características escolhido para representar o problema.

Pode-se também usar múltiplos classificadores, ao invés de apenas. Esta abordagem é chamada de sistemas com múltiplos classificadores (do inglês \textit{multiple classifier systems}), os quais podem ser compostos por classificadores do mesmo tipo (denominado \textit{Ensemble} de classificadores) ou de diferentes tipos. Existe uma grande variedade de algoritmos de aprendizagem de máquina propostos na literatura. Alguns dos mais utilizados são: classificadores estatísticos, redes neurais artificiais, árvores de decisão, máquinas de vetores de suporte (SVM), KNN, etc \cite{jain:1989}.

Amplamente utilizadas em algoritmos de classificação, as árvores de decisão são representações simples do conhecimento e um meio eficiente de construir classificadores que predizem classes baseadas nos valores de atributos de um conjunto de dados. As árvores de decisão consistem de nodos que representam os atributos; de arcos, provenientes destes nodos e que recebem os valores possíveis para estes atributos; e de nodos folha, que representam as diferentes classes de um conjunto de treinamento. Classificação, neste caso, é a construção de uma estrutura de árvore, que pode ser usada para classificar corretamente todos os objetos do conjunto de dados da entrada.

A partir de uma árvore de decisão é possível derivar regras. As regras são escritas considerando o trajeto do nodo raiz até uma folha da árvore. Estes dois métodos são geralmente utilizados em conjunto. Devido ao fato das árvores de decisão tenderem a crescer muito, de acordo com algumas aplicações, elas são muitas vezes substituídas pelas regras. Isto acontece em virtude das regras poderem ser facilmente modularizadas. Uma regra pode ser compreendida sem que haja a necessidade de se referenciar outras regras.

Uma árvore de decisão tem a função de particionar recursivamente um conjunto de treinamento, até que cada subconjunto obtido deste particionamento contenha casos de uma única classe. Para atingir esta meta, a técnica de árvores de decisão examina e compara a distribuição de classes durante a construção da árvore. O resultado obtido, após a construção de uma árvore de decisão, são dados organizados de maneira compacta, que são utilizados para classificar novos casos. A árvore de decisão não presume nenhum modelo estatístico a priori, sendo a divisão do espaço de atributos feita de acordo com as amostras provenientes do treinamento.


O algoritmo KNN (K-Nearest Neighbours, ou K vizinhos mais próximos) \cite{cover:1967} é um método de classificação baseada na proximidade de amostras de treino no espaço de características. É considerado um dos mais simples algoritmos de aprendizagem de máquina.

O processo de treinamento para esse algoritmo consiste em armazenar o vetor de característica e rótulos (classes) de cada amostra de treinamento em um espaço n-dimensional, onde n é o número de características de cada amostra. No processo de classificação de amostras não-rotuladas, a amostra é simplesmente projetada no espaço e é classificada de acordo com as k amostras mais próximas.

Quando k = 1, a amostra é simplesmente classificada de acordo com o rótulo de seu vizinho mais próximo no espaço de características. Quando k é maior que 1, a classificação se dá por um esquema de votação, onde a classe com as amostras vizinhas mais numerosas são consideradas como a classe da amostra. Por essa razão, em problemas bi-classe como o apresentado neste artigo devem possuir um k ímpar, para evitar empates. Em problemas multi-classe, ou seja, com mais de duas classes possíveis, empates podem acontecer mesmo considerando um número ímpar de vizinhos, de maneira que uma forma de desempate deve ser definida na implementação do algoritmo.

Diversas formas de calcular a distância entre amostras em um espaço n-dimensional podem ser utilizadas no kNN, dentre elas a distância Euclidiana:

\begin{center}
$\displaystyle d(x,y) = ||x - y|| = \sqrt{(x - y)*(x -y)}$ \\
$\displaystyle = \sqrt{\sum_{i=1}^{m}(x_i - y_i)^2} $
\end{center}


As máquinas de vetores de suporte, ou SVM (\textit{Support Vector Machine}) são classificadores baseados na teoria de aprendizagem estatística proposta por \cite{vapnik:1995}. A teoria é baseada em uma forte fundamentação matemática para estimação de dependências e previsão do aprendizado a partir de conjuntos de dados finitos. O objetivo do SVM é utilizar vetores de suporte para maximizar a distância das amostras aos hiperplano de separação entre as classes (figura \ref{fig:svm}), com isso simplificando o modelo e minimizando o erro de treinamento.

\begin{figure}[h!]
  \centering
  \includegraphics[scale=0.5]{imgs/svm}
  \caption{Amostras em um espaço bi-dimensional separadas por um hiperplano apoiado por vetores de suporte}
  \label{fig:svm}
\end{figure}

Tanto durante o desenvolvimento de uma solução, quanto após sua execução em ambiente de produção, é preciso aferir e quantificar o desempenho da técnica desenvolvida ou utilizada.

\section{Avaliação}\label{sec:avaliacao}

Avaliar o desempenho de uma técnica de aprendizagem é útil para determinar a qualidade do modelo criado, aferir se o modelo continua adequado e inspecionar se os atributos escolhidos são relevantes para a classificação das amostras.

Comumente, o percentual de acerto obtido na classificação das amostras é um importante parâmetro para medir o desempenho do modelo. Este parâmetro é conhecido como acurácia ou taxa de reconhecimento. O oposto da acurácia é conhecido como taxa de erro.

De grande importância também é a composição da matriz de confusão (tabela \ref{tab:matrixConfusao}). Nela pode-se avaliar como um modelo está se comportando em termos de falsos positivos (Um exemplo é classificado como pertencente à classe C, mas não é) e falsos negativos (um exemplo é atribuído a outra classe, mas deveria ser da classe C). A principal função desta matriz é a dar possibilidade de pensar sobre o custo dos erros, ou seja, mesmo que a taxa de acerto para o problema seja alta, uma ou mais classes do problema pode ter uma taxa de acerto bem abaixo do esperado.

\begin{table}[h]
  \centering
  \begin{tabular}{cccc}
  \multicolumn{2}{c}{\textbf{Resultado obtido}}                  &                               &                                              \\ \cline{1-2}
  \multicolumn{1}{|c|}{Classe A} & \multicolumn{1}{c|}{Classe B} &                               &                                              \\ \cline{1-3}
  \multicolumn{1}{|c|}{TP}       & \multicolumn{1}{c|}{FN}       & \multicolumn{1}{c|}{Classe A} & \multirow{2}{*}{\textbf{Resultado esperado}} \\ \cline{1-3}
  \multicolumn{1}{|c|}{FP}       & \multicolumn{1}{c|}{TN}       & \multicolumn{1}{c|}{Classe B} &                                              \\ \cline{1-3}
  \end{tabular}
  \caption{Um modelo de matriz de confusão}
  \label{tab:matrixConfusao}
\end{table}

Alguns valores podem ser obtidos através desta matriz. A própria acurácia do modelo pode ser obtida com a seguinte equação:

\begin{equation}
  Acurácia = \frac{TP+TN}{TP+FP+TN+FN}
\label{eq:acuracia}
\end{equation}

Ainda é possível obter a precisão e revocação. A precisão é o número de elementos relevantes recuperados dividido pelo número total de elementos recuperados (equação \ref{eq:precisao}) enquanto a revocação é definida como o número de elementos relevantes recuperados dividido pelo número total de elementos relevantes existentes, que deveriam ter sido recuperados (equação \ref{eq:revocacao}).

\begin{equation}
  Precisão = \frac{TP}{TP+FP}
\label{eq:precisao}
\end{equation}


\begin{equation}
  Revocação = \frac{TP}{TP+FN}
\label{eq:revocacao}
\end{equation}


\chapter{Trabalhos Relacionados}

% aerial image classification
% aerial image segmentation


O trabalho de \citeonline{dubuisson:2000} apresenta uma técnica de segmentação focada em imagens aéreas coloridas que realiza a segmentações separadamente por cor e textura, para no final unir as duas e chegar a uma segmentação final utilizando um algoritmo de classificação por máxima verosimilhança (\textit{Maximum Likelihood}). O objetivo do trabalho de \cite{dubuisson:2000} é atualização de mapas antigos a partir de imagens recentes, mas as técnicas, se comprovadamente úteis ao problema deste trabalho, podem ser aplicadas em uma pré-classificação de tipos de terreno.

De acordo com \cite{sadgal:2005}, o processamento de imagens digitais que representam cenas naturais requer elaboração substancial em todos os níveis: pré-processamento, segmentação, reconhecimento e interpretação. O trabalho apresentado sugere uma abordagem onde todas essas etapas acontecem em um único nível, e propõe um modelo de visão que tenta generalizar o reconhecimento de objetos utilizando categorização e cooperação.  A solução proposta combina processos estocásticos, dentre os quais Inferência Bayesiana, Campos Aleatórios de Markov, com métodos não estocásticos como Redes Neurais Artificiais. Esta diversidade de métodos é utilizada na segmentação e na extração de características de cores, texturas e formas, que depois são usadas na classificação dos objetos.

A pesquisa realizada por \cite{ahmadi:2013} tem como objetivo fazer segmentação semântica e classificação de imagens aéreas, pixel a pixel. Para tal, diversos classificadores e atributos das imagens são testados, chegando-se a conclusão de que o uso do algoritmo de KNN em características de cor e textura, mais precisamente o filtro de Gabor \cite{fogel:1989} dos canais de matiz, saturação e intensidade (HSV) de cada pixel, obtiveram os melhores resultados, com um desempenho computacional superior aos métodos de baseline.

O trabalho de \cite{ghiasi:2013} realiza segmentação e classificação de tipos de terreno em imagens aéreas através de dois passos: primeiramente a imagem é dividida em superpixels, utilizando a técnica de fluxos geométricos de \cite{levinshtein:2009}; posteriormente, cada superpixel tem suas características de textura e cor extraídas e é classificado através do algoritmo KNN. As características apontadas como mais úteis pelos autores são o Local Binary Pattern Histogram Fourier (LBP-HF) \cite{ahonen:2009} para informações de textura e histograma dos canais RGB para informações sobre cores. O artigo alega conseguir realizar o processo em tempo real, com precisão superior a 95\% em todas as classes utilizadas.

Este trabalho pretende adaptar ou enriquecer os métodos utilizados na literatura, aplicando-os especificamente à segmentação e classificação de imagens aéreas de floresta amazônica, que possui seus desafios característicos, visto que o tipo de terreno e vegetação apresentam padrões diferentes dos vários trabalhos realizados em áreas urbanas ou florestas temperadas.

\chapter{Metodologia}\label{cap:metodologia}

Neste capítulo é descrita a sequência de etapas que serão realizadas neste trabalho para que os objetivos de pesquisa sejam alcançados.

Em imagens aéreas como as utilizadas neste trabalho, é comum que as anomalias que indicam presença humana sejam relativamente grandes (pista de pouso, estradas, clareiras, etc), podendo ser definidas como uma região durante a segmentação da imagem. Mas muitas vezes, por outro lado, os objetos ou sinais são pequenos demais para serem encontrados no processo de segmentação (carros, barcos, cabanas, etc), o que torna interessante um nível adicional de classificação da imagem: uma que possa inspecionar o interior das regiões encontradas durante a segmentação e possam procurar por elementos anômalos mais sutis na imagem.

Por este motivo, a arquitetura para a solução proposta prevê uma etapa de segmentação das imagens, seguida por dois níveis de classificação: um responsável pela determinação do tipo de cada região encontrada na segmentação; e um outro responsável pela busca de anomalias internas destas mesmas regiões. A arquitetura geral da solução pode ser vista na figura \ref{fig:metDiagrama}.

\begin{figure}[h]
    \includegraphics[width=\textwidth]{imgs/arquitetura}
    \caption{Arquitetura da solução a ser desenvolvida para detecção de anomalias em imagens aéreas da floresta amazônica}
    \label{fig:metDiagrama}
\end{figure}

A primeira etapa do trabalho é a segmentação das imagens por tipo de terreno. Os métodos de segmentação de imagens considerados estado-da-arte serão aplicados à uma parte da base de imagens aéreas da floresta amazônica. Esta porção da base de imagens será manualmente segmentada por seres humanos e servirá de base de comparação para a segmentação realizada pelos métodos experimentados. Esta etapa deve determinar o método a ser utilizado na solução descrita pelo trabalho.

Posteriormente, as regiões encontradas na segmentação das imagens devem ter suas características extraídas para serem classificadas dentre diversas classes: floresta, vegetação rasteira, clareira, corpos d'água e elementos anômalos. É nesta etapa que se dará o primeiro nível de classificação. Para esta etapa, serão utilizados os métodos descritos na seção \ref{sec:trClassificacao}. Os resultados de cada método serão comparados à uma classificação realizada por seres humanos, permitindo que os métodos de aprendizagem supervisionados sejam utilizados. As regiões classificadas como "elementos anômalos" serão prontamente consideradas objetos de interesse. As demais regiões serão classificadas de acordo com seu tipo de terreno pertinente e serão investigadas com mais detalhes na próxima etapa.

A última etapa de classificação consiste em utilizar modelos de aprendizado específicos para cada tipo de terreno, para detectar objetos de interesse que não foram encontrados na etapa de segmentação, normalmente pequenos demais para serem definidos como uma região \textit{per se}, e que estão contidos em regiões maiores. Cada tipo de terreno possuirá um modelo de classificação específico, podendo se utilizar de um vetor de características diferente dos demais terrenos, para otimizar a precisão na detecção dos objetos de interesse. Nesta etapa, diversos classificadores serão treinados e testados para cada tipo de terreno e seus resultados serão comparados com uma classificação manual feita por seres humanos. Nesta etapa serão explorados classificadores unários e multi-classe a fim de obter o melhor resultado na deteção dos objetos de interesse e um classificador, com seus parâmetros e vetor de características, será selecionado para cada tipo de terreno.

Por fim, unindo a saída do primeiro nível de classificação, onde regiões de "elementos anômalos" podem ser encontradas, com a saída do segundo nível, onde os objetos de interesse serão procurados no interior das demais regiões, teremos uma saída única, apontando as regiões de uma imagem que possuam objetos de interesse na problemática em questão.

As sucessivas validações e comparações com a segmentação manual feita por seres humanos é importante para diminuir o erro em cada etapa, visto que imprecisões em um dos passos tende a propagar o erro nos passos seguintes. Se tomarmos como exemplo o princípio da solução, uma segmentação inadequada de uma imagem pode dificultar o primeiro nível de classificação, responsável pela classificação do tipo de região encontrada.

Durante o desenvolvimento do trabalho, artigos serão submetidos com os resultados intermediários, para divulgação e validação junto à comunidade científica. Por fim, a dissertação final do programa de mestrado será redigida, contendo o resultado dos experimentos, seus êxitos e problemas encontrados.

Experimentos preliminares foram realizados e resultados interessantes foram obtidos. O próximo capítulo descreve os experimentos realizados e discute os resultados preliminares encontrados.

\chapter{Experimentos e Resultados}\label{cap:experimentos}

\section{Base de dados}

A base de dados (imagens) utilizada advém do projeto GEOMA \cite{geoma}, financiado pelo Instituto de Pesquisas Espaciais (INPE). Trata-se de imagens coloridas, codificadas em JPEG e com 640 pixels de largura por 480 pixels de altura. A base é composta por fotografias ortogonais ao relevo (como pode ser visto na figura \ref{fig:amostra}), de altitudes variadas e tiradas a partir de aeronaves tripuladas, durante o trajeto entre diversas cidades da região amazônica.

No momento do início dos experimentos deste trabalho, estas imagens tiradas de aviões tripulados eram as únicas disponíveis publicamente. Podemos considerá-las válidas por terem sido tiradas em altitude de voo compatível com as missões de VANTs de vigilância, entre 900 e 1.100 metros do solo. Como este trabalho tem como objetivo utilizar apenas câmeras de espectro visível, são dispensáveis comparações de sensores com VANTs que eventualmente possuam sonar, câmeras infravermelho ou outros tipos de sensores.

\begin{figure}[h]
  \centering
  \begin{subfigure}[b]{0.3\textwidth}
    \includegraphics[width=\textwidth]{imgs/amostra1}
  \end{subfigure}%
  ~
  \begin{subfigure}[b]{0.3\textwidth}
    \includegraphics[width=\textwidth]{imgs/amostra2}
  \end{subfigure}%
  ~
  \begin{subfigure}[b]{0.3\textwidth}
    \includegraphics[width=\textwidth]{imgs/amostra3}
  \end{subfigure}%
  \caption{Amostras da base de dados}
  \label{fig:amostra}
\end{figure}

A base possui um total de 3.044 imagens, com dimensão total de 1,02 Gigabytes de dados. Cerca de 150 imagens foram utilizadas nos experimentos até o momento, visto que um grande esforço precisa ser despendido na classificação manual das imagens em todas as etapas do experimento, consumindo um tempo considerável.

Para criar uma referência (\textit{ground-truth}) para a segmentação das imagens da base de dados, uma ferramenta computacional executável em navegadores web foi construída (figura \ref{fig:manualseg}). A saída deste aplicativo é uma coleção, para cada imagem, de informações sobre bordas das regiões da imagem. Estas informações servirão de referência para avaliar o desempenho dos algoritmos de segmentação testados neste trabalho.

\begin{figure}[h]
  \centering
  \includegraphics[width=0.7\textwidth]{imgs/manualseg}
  \caption{Ferramenta para segmentação manual das imagens}
  \label{fig:manualseg}
\end{figure}

\section{Segmentação}

Um experimento de comparação entre diversos algoritmos de segmentação de imagens foi planejado. Cada imagem da base de dados foi segmentada por seres humanos, segmentações estas que constituíram a base de referência para o experimento. Cada imagem tem sua segmentação de referência consolidada a partir da segmentação manual de pelo menos 5 seres humanos.

\subsection{Protocolo experimental}

Para criação da segmentação manual de referência (\textit{ground-truth}), 31 voluntários, todos alunos de pós-graduação em informática ou áreas relacionadas, foram convidados a realizar a segmentação manual das imagens da base de dados, através de uma ferramenta\footnote{http://amazonsegmentation.ddns.net/} criada com esta finalidade. A ferramenta de software web criada consiste de uma tela onde o usuário pode desenhar sobre uma imagem. A ideia é que nesta imagem sejam circunscritas as bordas das regiões definidas pelo usuário, de acordo com instruções fornecidas pela ferramenta e que foram lidas obrigatoriamente por cada voluntário antes do início do experimento.

Em linhas gerais, as instruções orientam os voluntários a segmentar as imagens de acordo com a cobertura ou tipo de terreno, formação geológica ou vegetação, de acordo com os próprios critérios e granularidade que pareça mais adequada:

\begin{citacao}[english]
Your mission is to manually segment the given images as accurately as possible, according to your own judgment. The criteria here is terrain coverage. So, we would like to separate different vegetations, geological formations and human-made objects \cite{amazonsegmentation}
\end{citacao}

O conteúdo integral das instruções pode ser encontrado no apêndice \ref{apendice:instrucoesManualSeg}.

%TODO O que se faz para unir e consolidar as segmentações?
%TODO Este método foi proposto por quem?
%TODO Qual a base estatística?

\subsection{Métricas}


\subsection{Estudo comparativo}

Para determinar qual dos algoritmos de segmentação levantados na pesquisa bibliográfica teria melhor desempenho na base de dados utilizada neste trabalho, todos foram implementados ou adaptados. Os algoritmos foram testados em todas as imagens da base de dados do trabalho que possuíam segmentação manual consolidada pela ferramenta supracitada, usando estas como referência para avaliação das segmentações.

Para avaliar as diferentes segmentações, o método introduzido por \citeonline{martin:2001} foi utilizado. Este método consiste na sobreposição de regiões e comparação de informações sobre localização e orientação dos elementos de borda (\textit{edgels}).

Os resultados do experimento são apresentados na tabela \ref{tab:experimentoSegmentacao}:

\begin{table}[h]
\ABNTEXfontereduzida
\centering
\begin{tabulary}{\linewidth}{|L|R|R|}
\hline
\textbf{Algoritmo} & \textbf{Acurácia} & \textbf{Tempo/imagem} \\ \hline
Mean-shift  & 97,2\% & 6,39 s \\ \hline
JSEG        & 96,3\% & 14,82 s \\ \hline
MSEG        & 94,1\% & 0,33 s \\ \hline
SRM         & 91,9\% & 4,66 s \\ \hline
FSEG        & 81,4\% & 13,91 s \\ \hline
gPb-owt-ucm & 72,2\% & 237,32 s \\ \hline
\end{tabulary}
\caption{Comparação de métodos de segmentação em parte da base de imagens deste trabalho, ordenados por acurácia}
\label{tab:experimentoSegmentacao}
\end{table}

O algoritmo Mean-shift, além de ter conseguido a melhor precisão na nossa base de dados, conseguiu um bom desempenho de tempo. Com isso, pudemos determinar que ele será o algoritmo utilizado no trabalho, para a segmentação inicial das imagens.

O método JSEG conseguiu uma precisão bastante próxima do Mean-shift e será considerado uma alternativa, embora o tempo de segmentação deste algoritmo seja consideravelmente maior. A imagem \ref{fig:comparacaoSegmentacao} mostra a saída de alguns dos métodos testados, para fins de comparação visual.

\begin{figure}[htb]
	\centering
	\begin{minipage}[l]{0.51\linewidth}
		\begin{subfigure}[b]{\linewidth}
			\includegraphics[width=\linewidth]{imgs/seg_original}
			\caption{Imagem original}
		\end{subfigure}%
	\end{minipage}
	\begin{minipage}[r]{0.48\linewidth}
		\begin{subfigure}{.47\linewidth}
			\includegraphics[width=\linewidth]{imgs/seg_meanshift}
			\caption{Mean-shift}
		\end{subfigure}
		\begin{subfigure}{.47\linewidth}
			\includegraphics[width=\linewidth]{imgs/seg_mseg}
			\caption{MSEG}
		\end{subfigure}%
		\\
		\begin{subfigure}{.47\linewidth}
			\includegraphics[width=\linewidth]{imgs/seg_jseg}
			\caption{JSEG}
		\end{subfigure}
		\begin{subfigure}{.47\linewidth}
			\includegraphics[width=\linewidth]{imgs/seg_srm}
			\caption{SRM}
		\end{subfigure}%
	\end{minipage}
	\caption{Comparação visual de métodos de segmentação}
	\label{fig:comparacaoSegmentacao}
\end{figure}


Adicionalmente, experimentos foram realizados com técnicas de classificação. O objetivo era saber se poderíamos utilizar apenas uma etapa para realizar segmentação e o primeiro nível de classificação. Os resultados foram abaixo do que se consegue na etapa de segmentação isolada. Os resultados foram publicados no \textit{$10^th$ International Conference on Computer Vision Theory and Applications} e o artigo \cite{cavalcanti:2015} completo pode ser visto no apêndice \ref{cap:visapp2015}.

Para fins de comparação, os resultados deste segundo experimento são apresentados na tabela \ref{tab:experimentoArtigo}. O tempo de execução da classificação para cada imagem no artigo publicado ignora o tempo de extração de características da imagem. Para uma comparação correta com os métodos de segmentação experimentados anteriormente, esse tempo gasto em extração de características foi acrescido na tabela desta seção.

\begin{table}[h]
\ABNTEXfontereduzida
\centering
\begin{tabulary}{\linewidth}{|L|R|R|}
\hline
\textbf{Algoritmo} & \textbf{Acurácia} & \textbf{Tempo/imagem} \\ \hline
Random forest  & 96,0\% & 12,72 s \\ \hline
KNN            & 92,6\% & 22,89 s \\ \hline
Naive Bayes    & 92,8\% & 8,36 s \\ \hline
Decision tree  & 82,2\% & 14,49 s \\ \hline
\end{tabulary}
\caption{Comparação de métodos de classificação para segmentação das imagens em uma única etapa, ordenados por acurácia}
\label{tab:experimentoArtigo}
\end{table}


\section{Classificações de regiões}

Para o primeiro nível de classificação, que se resume a definir os tipos de terrenos das regiões encontradas na segmentação, um experimento com métodos bastante difundidos de aprendizagem de máquina foi realizado.

Utilizando as técnicas de K vizinhos mais próximos (KNN), máquinas de vetores de suporte (SVM) e árvores de decisão, um experimento foi conduzido em uma parte da base de dados. As mesmas 150 imagens do experimento de segmentação anteriormente apresentado foram utilizadas.

Primeiramente, todas as imagens foram segmentadas utilizando o método Mean-shift, por ter obtido o melhor desempenho no experimento de segmentação anterior. Cada região encontrada nesta segmentação foi devidamente classificada manualmente, e suas características foram extraídas. A partir desse ponto bases de dados de treinamento e validação supervisionadas foram criadas, onde cada região segmentada representava uma amostra de treinamento ou validação.

As características extraídas para cada região foram:
\begin{itemize}
	\item Cor média para os canais vermelho, verde e azul;
	\item Histograma dos canais vermelho, verde e azul;
	\item Tom de cinza médio;
	\item Histograma de tons de cinza.
\end{itemize}

Estas características foram escolhidas em primeira instância por serem de fácil extração, baixa dimensionalidade e simples depuração/verificação. Muitos dos trabalhos na literatura utilizam características de cor e histogramas em tons-de-cinza. Como um trabalho prospectivo, estas características foram consideradas bons pontos de partida.

O total de 7.111 amostras foram classificadas, onde 4.694 (66\%) foram utilizadas como amostras de treinamento e 2.417 (33\%) foram utilizadas como amostras de validação. O método SVM teve desempenho ligeiramente superior aos demais, como pode ser visto na tabela \ref{tab:experimentoClassificacao1}, que apresenta os resultados de acurácia, precisão, e revocação para os métodos analisados.

\begin{table}[h]
\ABNTEXfontereduzida
\centering
\begin{tabulary}{\linewidth}{|L|R|R|R|}
\hline
\textbf{Método} & \textbf{Acurácia} & \textbf{Precisão} & \textbf{Revocação} \\ \hline
SVM               & 89,19\% & 0,927 & 0,864 \\ \hline
KNN               & 88,51\% & 0,912 & 0,863 \\ \hline
Árvore de decisão & 87,09\% & 0,842 & 0,859 \\ \hline
\end{tabulary}
\caption{Comparação de métodos de classificação para regiões segmentadas das imagens, ordenados por acurácia}
\label{tab:experimentoClassificacao1}
\end{table}

Embora os resultados tenham sido promissores, ainda há bastante espaço para melhoria. Apenas características de cor e luminância foram exploradas, portanto, características de textura, morfologia e borda ainda precisam ser avaliadas.

\section{Próximos passos}

Para os próximos meses de pesquisa, diversas características serão extraídas e avaliadas para o primeiro nível de classificação:

\begin{itemize}
	\item Bordas;
	\item Textura;
	\item Morfologia;
	\item Bag of visual words
	\item Vizinhança de regiões.
\end{itemize}

Além disso, técnicas ensembles de classificadores também serão avaliados e comparados com os resultados atuais.

Para a fase final de classificação, será feito um levantamento de características e algoritmos de classificação para a detecção de elementos anômalos dentro das regiões segmentadas. Nesta fase serão explorados classificadores unitários (\textit{one-class classifiers}), visto que a literatura relata bons resultados com este tipo de classificador para detecção de anomalias e \textit{outliers}. Após estes experimentos, todo o fluxo de classificação será reavaliado, a fim de otimizar a taxa de aprendizado e generalização dos classificadores.

Por fim, o volume final da dissertação de mestrado será composto, levando em consideração os melhores resultados dos experimentos. Artigos também serão feitos e submetidos, refletindo partes relevantes da pesquisa e seus resultados.


\chapter{Resultados e conclusão}\label{cap:conclusao}

Neste trabalho foi apresentado um estudo comparativo entre diversas abordagens de aprendizagem supervisionada para solucionar o problema da detecção de elementos antrópicos em imagens aéreas da floresta amazônica.

Antes que os experimentos fossem concluídos, foi necessário confeccionar uma base de dados de segmentações manuais, realizada por diversos indivíduos. Este conjunto de dados foi essencial para a avaliação não apenas dos métodos de segmentação considerados estado-da-arte aplicados ao problema, mas também para a avaliação da consistência da segmentação humana na coleção de imagens em questão.

Dentre as técnicas de segmentação analisadas, o método SRM, que realiza o agrupamento de pixels em uma região de acordo com a similaridade da intensidade e cor dos pixels de regiões vizinhas, obteve erros de consistência global e local inferiores aos demais. É importante notar também que o algoritmo MSEG, que se utiliza de informações de cor e textura, obteve desempenho aproximado ao do SRM, mas com o tempo de execução cerca de quatorze vezes menor. Isto torna o algoritmo MSEG um forte candidato a substituir a técnica SRM em uma aplicação prática, em que o tempo de processamento da segmentação seja um fator de impacto.

Um subproduto do experimento que pretendeu analisar as técnicas de segmentação é uma base de cerca de 10.000 segmentos, devidamente identificados e rotulados. Este é um importante objetivo deste trabalho, pois, junto com as ferramentas de software desenvolvidas para facilitar o processo de classificação das amostras, esta base de dados possibilita que trabalhos futuros sejam realizados tanto na melhoria dos resultados obtidos quanto na reaplicação em outros temas.

Para a classificação dos segmentos e, por consequência, a detecção de elementos anômalos nas imagens, quatro abordagens de modelagem do problema de aprendizagem de máquina foram avaliadas.

A primeira delas utiliza classificadores multi-classe em uma base com rótulos de cinco classes (floresta, vegetação rasteira, água, terra e elementos antrópicos). Neste experimento o algoritmo \textit{K-Nearest Neighbor} (KNN) com $k=3$ obteve os melhores resultados de todo o trabalho. Apesar dos bons resultados, técnicas de aprendizado preguiçoso (\textit{lazy learning}) como o KNN não criam um modelo de aprendizado sucinto, necessitando de boa parte das amostras de treinamento para realizar a classificação de novas amostras, o que pode ser problemático em um ambiente real, com limitações de tempo de processamento e memória disponível.

Adicionalmente, um problema da abordagem com classificadores multi-classe é a pouca generalização do aprendizado sobre elementos antrópicos, visto que se trata de uma classe pouco previsível. Como os elementos antrópicos fazem parte do modelo de treinamento, novas amostras de objetos antrópicos cujas características se distanciem das amostras de elementos antrópicos da base de treinamento tendem a reduzir a precisão do modelo.

A segunda abordagem analisada, que utiliza classificadores binários, obteve desempenho acima do esperado. A expectativa era de que os modelos gerados por esta abordagem tivessem um baixo desempenho na classificação da classe de elementos antrópicos, uma vez que esta correspondia a apenas 0.3\% da base de dados. Esperava-se também que por, conta da re-classificação das amostras de segmentos que reagrupou as classes de floresta, vegetação rasteira, água e terra em uma única classe, o modelo gerado fosse pouco descritivo, hipótese que não pôde ser comprovada.

Neste experimento o algoritmo KNN também obteve o melhor desempenho para a classe de elementos antrópicos, desta vez se beneficiando da seleção de características feita através da técnica CFS, que utiliza a correlação entre os atributos do problema para reduzir a dimensão do vetor de características. Os resultados indicam que a seleção de atributos contribuiu significativamente para a simplificação do modelo gerado, tornando-o mais generalista e mais apto a descrever a classe de elementos naturais, resultado inverso ao experimento com classificadores multi-classe.

\todo[inline]{Falar do experimento de unários}

\todo[inline]{Falar do experimento de ensemble}

Como métrica para avaliação das abordagens de aprendizado ficou definido que a medida $F1$, que leva em consideração um balanço entre precisão e revocação, seria utilizada para julgar o desempenho dos algoritmos. Por conta da importância da classe, bem como forma de compensar o desequilíbrio na distribuição das classes do problema na base de dados de treinamento, a medida $F1$ da classe de elementos antrópicos foi o critério principal na avaliação dos métodos.

Utilizando este critério para ranquear as abordagens de aprendizado de acordo com seus respectivos melhores resultados, podemos afirmar que o melhor desempenho foi obtido pelo KNN como classificador de um problema multi-classe seguido por uma combinação de métodos unários, conforme exibido na tabela \ref{tab:resultadosFinais}.

\begin{table}[h]
\centering
\begin{tabulary}{\linewidth}{|L|R|R|R|}
\hline
\textbf{Abordagem}  & \textbf{Precisão} & \textbf{Revocação} & \textbf{F1} \\ \hline
Multi-classe (KNN)  & 0.889 & 0.727 & 0.800 \\ \hline
Conjunto de unários & 0.850 & 0.730 & 0.785 \\ \hline
Binária (KNN)       & 1.000 & 0.615 & 0.762 \\ \hline
Unária (OC-SVM)     & 0.842 & 0.615 & 0.711 \\ \hline
\end{tabulary}
\caption{Comparação de abordagens de aprendizagem de máquina para a classe de elementos antrópicos, ordenados pela medida F1}
\label{tab:resultadosFinais}
\end{table}

\todo[inline]{Falar do quão promissor foi a abordagem de ensemble de classificadores unários}

\todo[inline]{Falar que tudo é público e está disponível no github}

\section{Trabalhos futuros}

\todo[inline]{Continuar explorando ensembles}
\todo[inline]{Explorar novos métodos de apuração e novos algoritmos de oneclass}


\bibliography{dissertacao}

\apendices

\partapendices

\chapter{Artigo VISAPP 2015}\label{cap:apendice}

\includepdf[pages={-},addtotoc={1,chapter,1,A comparison on supervised machine learning classification techniques for semantic segmentation of aerial images of rain forest regions,anexo1}]{articles/visapp2015.pdf}

\end{document}
