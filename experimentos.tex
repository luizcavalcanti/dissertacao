\chapter{Experimentos e Resultados Preliminares}\label{cap:experimentos}

\section{Base de dados}

A base de dados (imagens) utilizada advém do projeto GEOMA \cite{geoma}, financiado pelo Instituto de Pesquisas Espaciais (INPE). Trata-se de imagens coloridas, codificadas em JPEG e com 640 pixels de largura por 480 pixels de altura. A base é composta por fotografias ortogonais ao relevo (como pode ser visto na figura \ref{fig:amostra}), de altitudes variadas e tiradas a partir de aeronaves tripuladas, durante o trajeto entre diversas cidades da região amazônica.

No momento do início dos experimentos deste trabalho, estas imagens tiradas de aviões tripulados eram as únicas disponíveis publicamente. Podemos considerá-las válidas por terem sido tiradas em altitude de vôo compatível com as missões de VANTs de vigilância, entre 900 e 1.100 metros do solo. Como este trabalho tem como objetivo utilizar apenas câmeras de espectro visível, são dispensáveis comparações de sensores com VANTs que eventualmente possuam sonar, câmeras infravermelho ou outros tipos de sensores.

\begin{figure}[h]
  \centering
  \begin{subfigure}[b]{0.3\textwidth}
    \includegraphics[width=\textwidth]{imgs/amostra1}
  \end{subfigure}%
  ~
  \begin{subfigure}[b]{0.3\textwidth}
    \includegraphics[width=\textwidth]{imgs/amostra2}
  \end{subfigure}%
  ~
  \begin{subfigure}[b]{0.3\textwidth}
    \includegraphics[width=\textwidth]{imgs/amostra3}
  \end{subfigure}%
  \caption{Amostras da base de dados}
  \label{fig:amostra}
\end{figure}

A base possui um total de 3.044 imagens, com dimensão total de 1,02 Gigabytes de dados. Cerca de 150 imagens foram utilizadas nos experimentos até o momento, visto que um grande esforço precisa ser despendido na classificação manual das imagens em todas as etapas do experimento, consumindo um tempo considerável.

Para facilitar a classificação manual foi construída uma ferramenta gráfica que permite ao usuário segmentar e classificar as regiões das imagens de acordo com as classes disponíveis (figura \ref{fig:visualClassifier}). A saída deste aplicativo é uma coleção, para cada imagem, de informações sobre bordas das regiões e a classificação de cada pixel. Posteriormente essas informações serão confrontadas com o resultado da segmentação e primeiro nível de classificação da solução.

\begin{figure}[h]
  \centering
  \includegraphics[width=0.7\textwidth]{imgs/visualClassifier}
  \caption{Ferramenta para segmentação e classificação manual das imagens em regiões}
  \label{fig:visualClassifier}
\end{figure}

\section{Segmentação}

Para determinar qual dos algoritmos de segmentação levantados na pesquisa bibliográfica teria melhor desempenho na base de dados utilizada neste trabalho, todos foram implementados ou adaptados. Os algoritmos foram testados no conjunto inicial de 150 imagens e comparados com a segmentação manual realizada nas mesmas imagens.

Para avaliar as diferentes segmentações, o método introduzido por \citeonline{martin:2001} foi utilizado. Este método consiste na sobreposição de regiões e comparação de informações sobre localização e orientação dos elementos de borda (\textit{edgels}).

Os resultados do experimento são apresentados na tabela \ref{tab:experimentoSegmentacao}:

\begin{table}[h]
\ABNTEXfontereduzida
\centering
\begin{tabulary}{\linewidth}{|L|R|R|}
\hline
\textbf{Algoritmo} & \textbf{Precisão} & \textbf{Tempo/imagem} \\ \hline
Mean-shift  & 97,2\% & 6,39 s \\ \hline
JSEG        & 96,3\% & 14,82 s \\ \hline
MSEG        & 94,1\% & 0,33 s \\ \hline
SRM         & 91,9\% & 4,66 s \\ \hline
FSEG        & 81,4\% & 13,91 s \\ \hline
gPb-owt-ucm & 72,2\% & 237,32 s \\ \hline
\end{tabulary}
\caption{Comparação de métodos de segmentação em parte da base de imagens deste trabalho, ordenados por precisão}
\label{tab:experimentoSegmentacao}
\end{table}

O algoritmo Mean-shift, além de ter conseguido a melhor precisão na nossa base de dados, conseguiu um bom desempenho de tempo. Com isso, pudemos determinar que ele será o algoritmo utilizado no trabalho, para a segmentação inicial das imagens.

O método JSEG conseguiu uma precisão bastante próxima do Mean-shift e será considerado uma alternativa, embora o tempo de segmentação deste algoritmo seja consideravelmente maior. A imagem \ref{fig:comparacaoSegmentacao} mostra a saída de alguns dos métodos testados, para fins de comparação visual.

\begin{figure}[htb]
	\centering
	\begin{minipage}[l]{0.51\linewidth}
		\begin{subfigure}[b]{\linewidth}
			\includegraphics[width=\linewidth]{imgs/seg_original}
			\caption{Imagem original}
		\end{subfigure}%
	\end{minipage}
	\begin{minipage}[r]{0.48\linewidth}
		\begin{subfigure}{.47\linewidth}
			\includegraphics[width=\linewidth]{imgs/seg_meanshift}
			\caption{Mean-shift}
		\end{subfigure}
		\begin{subfigure}{.47\linewidth}
			\includegraphics[width=\linewidth]{imgs/seg_mseg}
			\caption{MSEG}
		\end{subfigure}%
		\\
		\begin{subfigure}{.47\linewidth}
			\includegraphics[width=\linewidth]{imgs/seg_jseg}
			\caption{JSEG}
		\end{subfigure}
		\begin{subfigure}{.47\linewidth}
			\includegraphics[width=\linewidth]{imgs/seg_srm}
			\caption{SRM}
		\end{subfigure}%
	\end{minipage}
	\caption{Comparação visual de métodos de segmentação}
	\label{fig:comparacaoSegmentacao}
\end{figure}


Adicionalmente, experimentos foram realizados com técnicas de classificação. O objetivo era saber se poderíamos utilizar apenas uma etapa para realizar segmentação e o primeiro nível de classificação. Os resultados foram abaixo do que se consegue na etapa de segmentação isolada. Os resultados foram publicados no \textit{$10^th$ International Conference on Computer Vision Theory and Applications} e o artigo \cite{cavalcanti:2015} completo pode ser visto no apêndice \ref{cap:apendice}.

Para fins de comparação, os resultados deste segundo experimento são apresentados na tabela \ref{tab:experimentoArtigo}. O tempo de execução da classificação para cada imagem no artigo publicado ignora o tempo de extração de características da imagem. Para uma comparação correta com os métodos de segmentação experimentados anteriormente, esse tempo foi acrescido na tabela desta seção.

\begin{table}[h]
\ABNTEXfontereduzida
\centering
\begin{tabulary}{\linewidth}{|L|R|R|}
\hline
\textbf{Algoritmo} & \textbf{Precisão} & \textbf{Tempo/imagem} \\ \hline
Random forest  & 96,0\% & 12,72 s \\ \hline
KNN            & 92,6\% & 22,89 s \\ \hline
Naive Bayes    & 92,8\% & 8,36 s \\ \hline
Decision tree  & 82,2\% & 14,49 s \\ \hline
\end{tabulary}
\caption{Comparação de métodos de classificação para segmentação das imagens em uma única etapa, ordenados por precisão}
\label{tab:experimentoArtigo}
\end{table}


\section{Classificações de regiões}

%TODO realizar experimentos com KNN
%TODO realizar experimentos com SVM
%TODO realizar experimentos com árvores

\section{Próximos passos}

%TODO falar dos experimentos futuros com one-class classifiers