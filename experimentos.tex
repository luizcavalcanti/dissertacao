\chapter{Experimentos e Resultados Preliminares}

\section{Base de dados}

A base de dados (imagens) utilizada advém do projeto GEOMA \textbf{CITAR GEOMA}, financiado pelo Instituto de Pesquisas Espaciais (INPE). Trata-se de imagens coloridas, codificadas em JPEG descomprimido e de 640 pixels de largura por 480 pixels de altura. A base é composta de fotografias ortogonais ao relevo (como pode ser visto na figura \ref{fig:amostra}), de altitudes variadas e tiradas a partir de aeronaves tripuladas, durante o trajeto entre diversas cidades da região amazônica.

\begin{figure}
  \centering
  \begin{subfigure}[b]{0.3\textwidth}
    \includegraphics[width=\textwidth]{imgs/amostra1}
  \end{subfigure}%
  ~
  \begin{subfigure}[b]{0.3\textwidth}
    \includegraphics[width=\textwidth]{imgs/amostra2}
  \end{subfigure}%
  ~
  \begin{subfigure}[b]{0.3\textwidth}
    \includegraphics[width=\textwidth]{imgs/amostra3}
  \end{subfigure}%
  \caption{Amostras da base de dados}
  \label{fig:amostra}
\end{figure}

A base possui um total de 3.044 imagens, com dimensão total de 1,02 Gigabytes de dados, e foi utilizada apenas parcialmente durante os experimentos, até o momento.

\section{Análise de características}

TO-DO

\section{Segmentação e classificação de terreno}

TO-DO

\section{Detecção de anomalias}

TO-DO

\section{Resultados preliminares}

TO-DO

\section{Próximos passos}

TO-DO