\chapter{Trabalhos Relacionados}

% aerial image classification
% aerial image segmentation


O trabalho de \citeonline{dubuisson:2000} apresenta uma técnica de segmentação focada em imagens aéreas coloridas que realiza a segmentações separadamente por cor e textura, para no final unir as duas e chegar a uma segmentação final utilizando um algoritmo de classificação por máxima verosimilhança (\textit{Maximum Likelihood}). O objetivo do trabalho de \cite{dubuisson:2000} é atualização de mapas antigos a partir de imagens recentes, mas as técnicas, se comprovadamente úteis ao problema deste trabalho, podem ser aplicadas em uma pré-classificação de tipos de terreno.

De acordo com \cite{sadgal:2005}, o processamento de imagens digitais que representam cenas naturais requer elaboração substancial em todos os níveis: pré-processamento, segmentação, reconhecimento e interpretação. O trabalho apresentado sugere uma abordagem onde todas essas etapas acontecem em um único nível, e propõe um modelo de visão que tenta generalizar o reconhecimento de objetos utilizando categorização e cooperação.  A solução proposta combina processos estocásticos, dentre os quais Inferência Bayesiana, Campos Aleatórios de Markov, com métodos não estocásticos como Redes Neurais Artificiais. Esta diversidade de métodos é utilizada na segmentação e na extração de características de cores, texturas e formas, que depois são usadas na classificação dos objetos.

A pesquisa realizada por \cite{ahmadi:2013} tem como objetivo fazer segmentação semântica e classificação de imagens aéreas, pixel a pixel. Para tal, diversos classificadores e atributos das imagens são testados, chegando-se a conclusão de que o uso do algoritmo de KNN em características de cor e textura, mais precisamente o filtro de Gabor \cite{fogel:1989} dos canais de matiz, saturação e intensidade (HSV) de cada pixel, obtiveram os melhores resultados, com um desempenho computacional superior aos métodos de baseline.

O trabalho de \cite{ghiasi:2013} realiza segmentação e classificação de tipos de terreno em imagens aéreas através de dois passos: primeiramente a imagem é dividida em superpixels, utilizando a técnica de fluxos geométricos de \cite{levinshtein:2009}; posteriormente, cada superpixel tem suas características de textura e cor extraídas e é classificado através do algoritmo KNN. As características apontadas como mais úteis pelos autores são o Local Binary Pattern Histogram Fourier (LBP-HF) \cite{ahonen:2009} para informações de textura e histograma dos canais RGB para informações sobre cores. O artigo alega conseguir realizar o processo em tempo real, com precisão superior a 95\% em todas as classes utilizadas.

Este trabalho pretende adaptar ou enriquecer os métodos utilizados na literatura, aplicando-os especificamente à segmentação e classificação de imagens aéreas de floresta amazônica, que possui seus desafios característicos, visto que o tipo de terreno e vegetação apresentam padrões diferentes dos vários trabalhos realizados em áreas urbanas ou florestas temperadas.