\chapter{Trabalhos Relacionados}

% aerial image classification
% aerial image segmentation

Os trabalhos relacionados apresentados nesta seção foram divididos primeiramente por assunto: de início, nos concentramos nos trabalhos publicados em segmentação de imagens que são considerados estado da arte, uma vez que esta é o primeiro desafio técnico a ser enfrentado pelo trabalho proposto. Depois, fazemos o levantamento de trabalhos relevantes sobre classificação de imagens aéreas.

Ambas as seções a seguir apresentam os trabalhos relacionados em ordem própria. A seção \ref{sec:trSegmentacao} apresenta os trabalhos sobre segmentação de imagens de acordo com a ordem apresentada no trabalho de comparação realizado por \citeonline{arbelaez:2011}, explanado com mais detalhes sem seguida. A seção \ref{sec:trClassificacao} apresenta os trabalhos relacionados sobre classificação de imagens aéreas em ordem cronológica de publicação.

Cara artigo é descrito com suas características, vantagens e desvantagens. No fim de cada seção, é apresentada uma tabela de comparação entre os métodos utilizados na literatura.

\section{Segmentação de imagens}\label{sec:trSegmentacao}

%\textbf{palavras-chave}: aerial image segmentation, terrain classification, terrain segmentation

Os trabalhos relacionados nesta seção foram escolhidos por terem obtido os maiores índices de precisão e revocação em uma base de dados de imagens naturais bastante conhecida pela comunidade de processamento de imagens, \textit{Berkeley Segmentation Data Set and Benchmarks 500}, também conhecida como BSDS500. Uma comparação entre os mais promissores algoritmos de segmentação de imagens foi realizada por \citeonline{arbelaez:2011}, consistindo na segmentação das imagens presentes no BSDS500 manualmente por seres humanos, e depois segmentadas pelos diversos algoritmos testados. A precisão e revocação de cada algoritmo são obtidas através de comparação com a segmentação manual.

Para reduzir ainda mais o número de algoritmos a serem inspecionados para este trabalho, a revisão de \citeonline{yuan2:2013} compara os mesmos algoritmos que \citeonline{arbelaez:2011} investigaram, mas dessa vez em uma base de imagens aéreas. A metodologia é a mesma usada no BSDS500, mas com uma base especializada. Os algoritmos com os melhores resultados foram selecionados para serem utilizados na construção deste trabalho de pesquisa, e coincidem com os resultados do trabalho de \citeonline{arbelaez:2011}. Os seis algoritmos selecionados serão discutidos a seguir.

O trabalho de \citeonline{deng:2001} trata de um algoritmo chamado JSEG, que obtém a segmentação da imagem em duas etapas. A primeira é a  quantização das cores da imagem em diversas classes. Baseado nessas cores quantizadas, a segunda etapa computa o valor de uma variável \textit{J} indicando a intensidade das bordas e utiliza um método de crescimento de região para segmentar a imagem baseada neste valor \textit{J}. O algoritmo ainda permite que o utilizador especifique o tamanho da janela para computar o valor de \textit{J}, o que torna o método bastante flexível para imagens de naturezas diferentes. Em imagens aéreas de escala considerável, como as utilizadas neste trabalho, pode-se usar uma janela diminuta, já que os detalhes importantes podem ser bem pequenos. Para diminuir super-segmentação, os segmentos encontrados na segunda etapa são fundidos de acordo com seus histogramas coloridos.

A abordagem utilizando mean-shift desenvolvida por \citeonline{comaniciu:2002} oferece uma ferramenta interessante para resolver o problema de segmentação de imagens. O algoritmo computa vetores de mean-shift iterativamente para mapear pixels para o domínio espacial e de cores do centro de seus agrupamentos (\textit{clusters}). Após a convergência, os \textit{clusters} são fundidos de acordo com parâmetros de similaridade. Parâmetros como largura de banda espacial, de cores e o tamanho do menor \textit{cluster}.

O algoritmo MSEG, descrito por \citeonline{felzenszwalb:2004}, é amplamente usado pela comunidade de sensoriamento remoto. No MSEG, o aumento na heterogeneidade no momento da junção de um par de segmentos é computado como uma soma ponderada de medidas de coloração e morfologia. O procedimento de junção é realizado iterativamente, e  junta os pares de segmento que resultam no menor aumento de heterogeneidade possível, até que a soma exceda um limiar, que pode ser parametrizado pelo utilizador. Sob a óptica de grafos, os pixels são tratados como nodos e os pesos de suas arestas representam a diferença entre características entre os pixels. Um segmento corresponde a um componente conectado nesse grafo. Um parâmetro é utilizado para definir a escala observada na segmentação, o que influi no tamanho e número dos segmentos resultantes.

O algoritmo de \textit{Statistical Region Merging} (SRM) publicado por \citeonline{nock:2004} utiliza um procedimento simples de junção acompanhado de uma operação de ordenação para segmentar imagens com eficiência. Duas regiões são unidas se os valores médios dos pixels das duas regiões estão mais próximos que um limiar previamente definido. A coesão da segmentação pode ser controlada por um parâmetro definido pelo usuário.

O trabalho de \citeonline{yuan:2013} apresenta um algoritmo chamado \textit{Factorization-based segmentation} (FSEG). O FSEG primeiramente computa o histograma espectral para cada pixel da imagem, que é um amalgama de diversas respostas aos filtros em uma janela local. Baseado na proposta de que cada característica pode ser aproximada por uma combinação linear de diversas características representativas e suas combinações ponderadas. Por fim, um pixel é dito pertencente à região com maior peso. O algoritmo FSEG utiliza decomposição de valores singulares e fatorização de matrizes não-negativas para aumentar a eficiência computacional da segmentação, o que é atrativo quando o tempo de processamento é um fator importante.

O algoritmo gPb-owt-ucm introduzido por \citeonline{arbelaez:2011} realiza segmentação em várias etapas. Primeiramente a técnica combina informações de intensidade, textura e coloração para computar vetores que servirão como  detectores de contorno. Posteriormente, uma técnica de \textit{watershed} é aplicada à saída do detector de contornos para produzir uma segmentação hierárquica da imagem. Uma vantagem importante desta técnica é a possibilidade de ajustar a escala de segmentação, portanto o usuário pode escolher uma escala que mais se adequa ao tipo de imagem do problema, evitando segmentação excessiva da imagem.

A tabela \ref{tab:sumarioSegmentacao} exibe uma comparação das características das diversas técnicas de segmentação de imagens relacionados levantadas neste trabalho.

\begin{table}[h]
\ABNTEXfontereduzida
\centering
\begin{tabulary}{\linewidth}{|L|L|C|C|}
\hline
\textbf{Técnica} & \textbf{Características} & \textbf{Prec. bordas} & \textbf{Prec. regiões } \\ \hline
Manual      & Não aplicável    & 69\% & 84\% \\ \hline
gPb-owt-ucm & Cor, borda       & 65\% & 69\% \\ \hline
FSEG        & Textura          & 61\% & 66\% \\ \hline
SRM         & Cor, intensidade & 60\% & 60\% \\ \hline
JSEG        & Cor, borda       & 56\% & 66\% \\ \hline
MSEG        & Cor, morfologia  & 57\% & 50\% \\ \hline
Mean-shift  & Cor, posição     & 58\% & 48\% \\ \hline
\end{tabulary}
\caption{Comparação entre os trabalhos sobre segmentação de imagens, ordenados por desempenho decrescente}
\label{tab:sumarioSegmentacao}
\end{table}

Tanto na revisão feita por \citeonline{arbelaez:2011} com a base de imagens naturais BSD500 quanto na comparação feita por \citeonline{yuan2:2013} em uma base de imagens aéreas, o algoritmo gPb-owt-ucm de \citeonline{arbelaez:2011} possui um desempenho superior aos demais algoritmos avaliados. Vale notar que em ambos os trabalhos, todos os algoritmos têm desempenho sensivelmente inferior à segmentação manual feita por seres humanos. 

Uma vez que o trabalho aqui apresentado se propõe a, em uma primeira etapa, não apenas segmentar, mas também classificar os diferentes tipos de terreno em uma imagem aérea, um levantamento bibliográfico foi feito sobre classificação de imagens aéreas.

\section{Classificação de imagens aéreas}\label{sec:trClassificacao}

%\textbf{palavras-chave}: aerial image classification
Os trabalhos relacionados sobre classificação de imagens aéreas foram selecionados de acordo com suas semelhanças com o trabalho proposto neste documento. Todos tratam de imagens aéreas ortogonais, ou seja, com inclinação de aproximadamente 90\degree em relação ao solo, em cenas naturais e com bases de dados com forte presença de vegetação. Os trabalhos relacionados expressam diferentes formas de classificação das imagens: classificação de pixels individuais, blocos ou superpixels.

Todos os trabalhos descritos a seguir possuem imagens em escala similar às imagens de VANTs, baseando-se na altitude regular de operação desses equipamentos. O fato de terem sido coletadas com VANTs ou não, é de pouca importância para efeitos de comparação, já que os aspectos técnicos das imagens são similares.

O trabalho de \citeonline{dubuisson:2000} apresenta uma técnica de segmentação focada em imagens aéreas coloridas que realiza a segmentações separadamente por cor e textura, para no final unir as duas e chegar a uma segmentação final utilizando um algoritmo de classificação por máxima verosimilhança (\textit{Maximum Likelihood}). O objetivo do trabalho de \citeonline{dubuisson:2000} é atualização de mapas antigos a partir de imagens recentes, mas pode facilmente ser utilizado na classificação por de tipos de terreno.

De acordo com \citeonline{sadgal:2005}, o processamento de imagens digitais que representam cenas naturais requer elaboração substancial em todos os níveis: pré-processamento, segmentação, reconhecimento e interpretação. O trabalho apresentado sugere uma abordagem onde todas essas etapas acontecem em um único nível, e propõe um modelo de visão que tenta generalizar o reconhecimento de objetos utilizando categorização e cooperação.  A solução proposta combina processos estocásticos, dentre os quais Inferência Bayesiana, Campos Aleatórios de Markov, com métodos não-estocásticos como Redes Neurais Artificiais. Esta diversidade de métodos é utilizada na segmentação e na extração de características de cores, texturas e formas, que depois são usadas na classificação dos objetos. Uma vantagem importante deste método é a possibilidade de paralelizar o processo de classificação, uma vez que as diversas técnicas são fundidas no final do processo, ao invés de serem aplicadas em cascata. Embora os resultados pareçam ser satisfatórios em imagens naturais, pouco é dito sobre como o processo de fusão de classificadores é feito, e nenhuma implementação ou base de imagens está disponível publicamente.

A pesquisa realizada por \citeonline{ahmadi:2013} tem como objetivo fazer segmentação e classificação de imagens aéreas, pixel a pixel. Para tal, diversos classificadores e atributos das imagens são testados, chegando-se a conclusão de que o uso do algoritmo de KNN em características de cor e textura, mais precisamente o filtro de Gabor \cite{fogel:1989} dos canais de matiz, saturação e intensidade (HSV) de cada pixel, obtiveram os melhores resultados, com um desempenho computacional superior aos métodos de baseline.

O trabalho de \citeonline{ghiasi:2013} realiza segmentação e classificação de tipos de terreno em imagens aéreas através de dois passos: primeiramente a imagem é dividida em superpixels, utilizando a técnica de fluxos geométricos de \citeonline{levinshtein:2009}; posteriormente, cada superpixel tem suas características de textura e cor extraídas e é classificado através do algoritmo KNN. As características apontadas como mais úteis pelos autores são o Local Binary Pattern Histogram Fourier (LBP-HF) \cite{ahonen:2009} para informações de textura e histograma dos canais RGB para informações sobre cores. O artigo alega conseguir realizar o processo em tempo real, com precisão superior a 95\% em todas as classes utilizadas.

A tabela \ref{tab:sumarioClassificacao} apresenta um sumário dos trabalhos levantados nesta seção, ordenados conforme aparição.

\begin{table}[h]
\ABNTEXfontereduzida
\centering
\begin{tabulary}{\linewidth}{|L|L|L|L|C|}
\hline
\textbf{Trabalho} &  \textbf{Aprendizado} & \textbf{Amostra} & \textbf{Problema investigado} &  \textbf{Precisão} \\ \hline
Dub-Jolly & Máxima verosimilhança & Pixel      & Atualização de mapas           & 91,86\% \\ \hline
Sadgal    & Redes neurais         & Blocos     & Classificação de terreno       & -       \\ \hline
Ahmadi    & KNN                   & Pixel      & Classificação de terreno       & 82,23\% \\ \hline
Ghiasi    & KNN                   & Superpixel & Busca por objetos de interesse & 95\%    \\ \hline
\end{tabulary}
\caption{Comparação entre os trabalhos sobre classificação de imagens aéreas}
\label{tab:sumarioClassificacao}
\end{table}

Este trabalho pretende adaptar ou enriquecer os métodos utilizados na literatura, aplicando-os especificamente à detecção de anomalias em imagens aéreas da floresta amazônica, que possui seus desafios característicos, visto que o tipo de terreno e vegetação apresentam padrões diferentes dos vários trabalhos realizados em áreas urbanas ou florestas temperadas. Os trabalhos encontrados relacionados à floresta amazônica comumente se utilizam de outros tipos de sensores como RADAR e LIDAR, além de câmera de espectro visível, como pode ser visto nos trabalhos de \citeonline{linhares:2014} e \citeonline{santos:2005}. Estes trabalhos também tendem a se preocupar com dados temporais, como avanço do desmatamento e levantamento de grandes áreas.

O diferencial deste trabalho em relação aos demais reside na organização e ordenação das etapas de aprendizado, realizando a classificação em dois níveis: primeiramente, cada tipo de terreno encontrado na imagem é segmentado e classificado; posteriormente, métodos de classificação específicos para cada tipo de terreno são empregados para a detecção de anomalias. Mais detalhes sobre o método proposto serão discutidos no capítulo \ref{cap:metodologia}.